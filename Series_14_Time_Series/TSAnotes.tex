\documentclass{beamer}

\usepackage{framed}
\usepackage{amsmath}
\begin{document}
%===================================================%
<p>
\frametitle{Time Series Analysis}
\huge
\[ \mbox{ Time Series Analysis
and Business
Forecasting} \]


%===================================================%
<p>
\frametitle{Time Series Analysis}
\textbf{PART A: THE CLASSICAL TIME SERIES MODEL}
\begin{itemize}
\item A time series is a set of observed values, such as production or sales data, for a sequentially ordered series
of time periods. 
\item Examples of such data are sales of a particular product for a series of months and the number of
workers employed in a particular industry for a series of years. 
\item A time series is portrayed graphically by a line
graph, with the time periods represented on the horizontal axis and time series values
represented on the vertical axis.
\end{itemize}
%% LINE GRAPH

%===================================================%
<p>
\frametitle{Time Series Analysis}
\textbf{EXAMPLE 1.}
\begin{itemize}
\item  Figure 16-1 is a line graph that portrays the annual dollar sales for a graphics software company (fictional)
that was incorporated in 1990. 
\item As can be observed, a peak in annual sales was achieved in 1995, followed by two years of
declining sales that culminated in the trough in 1997, which was then followed by increasing levels of sales during the
final three years of the reported time series values.
\end{itemize}

%===================================================%
<p>
\frametitle{Time Series Analysis}
\begin{itemize}
\item Time series analysis is the procedure by which the time-related factors that influence the values observed in
the time series are identified and segregated. 
\item Once identified, they can be used to aid in the interpretation of
historical time series values and to forecast future time series values. 
\end{itemize}

%===================================================%
<p>
\frametitle{Time Series Analysis}
The classical approach to time series
analysis identifies four such influences, or, components:
\begin{itemize}
\item[(1)] Trend (T): The general long-term movement in the time series values (Y) over an extended period of years.
\item[(2)] Cyclical fluctuations (C): Recurring up and down movements with respect to trend that have a duration of
several years.
\item[(3)] Seasonal variations (S): Up and down movements with respect to trend that are completed within a year and
recur annually. Such variations typically are identified on the basis of monthly or quarterly data.
\item[(4)] Irregular variations (I): The erratic variations from trend that cannot be ascribed to the cyclical or seasonal
influences.

%===================================================%
<p>
\frametitle{Time Series Analysis}
The model underlying classical time series analysis is based on the assumption that for any designated
period in the time series the value of the variable is determined by the four components defined above, and,
furthermore, that the components have amultiplicative relationship. Thus, where Y represents the observed time
series value,
\[Y = T \times  C \times  S \times  I \]
The model represented by (16.1) is used as the basis for separating the influences of the various components
that affect time series values, as described in the following sections of this chapter.


%===================================================%
<p>
\frametitle{Time Series Analysis}
\textbf{Part B: TREND ANALYSIS}
\begin{itemize}
\item Because trend analysis is concerned with the long-term direction of movement in the time series, such
analysis generally is performed using annual data. 
\item Typically, at least 15 or 20 years of data should be used, so
that cyclical movements, involving several years duration are not taken to be indicative of the overall trend of
the time series values.
\end{itemize}

%===================================================%
<p>
\frametitle{Time Series Analysis}
\begin{itemize}
\item The method of least squares is the most frequent basis used for identifying the trend
component of the time series by determining the equation for the best-fitting trend line. 
\item Note that statistically
speaking, a trend line is not a regression line, since the dependent variable Y is not a random variable, but,
rather, a series of historical values. 
\item Further, there can be only one historical value for any given time period (not
a distribution of values) and the values associated with adjoining time periods are likely to be dependent, rather
than independent. 
\end{itemize}

%===================================================%
<p>
\frametitle{Time Series Analysis}
Nevertheless, the least-squares method is a convenient basis for determining the trend
component of a time series. When the long-term increase or decrease appears to follow a linear trend, the
equation for the trend line values, with X representing the year, is
YT ¼ b0 þ b1X (16:2)
As explained in Section 14.3, the b0 in (16.2) represents the point of intersection of the trend line with the Y
axis, whereas the b1 represents the slope of the trend line. Where X is the year and Y is the observed time series
value, the formulas for determining the values of b0 and b1 for the linear trend equation are
%% EQUATION

%===================================================%
<p>
\frametitle{Time Series Analysis}
See Problem 16.1 for the determination of a linear trend equation.
In the case of nonlinear trend, a type of trend curve often found to be useful is the exponential trend curve.
A typical exponential trend curve is one that reflects a constant rate of growth during a period of years, as might
CHAP. 16] TIME SERIES ANALYSIS AND BUSINESS FORECASTING 297
apply to the sales of personal computers during the 1980s. See Fig. 16-2(a). An exponential curve is so named
because the independent variable X is the exponent of b1 in the general equation:
YT ¼ b0 bX
1 (16:5)
where b0 ¼ value of YT in Year 0
b1 ¼ rate of growth (e.g., 1:10 ¼ 10% rate of growth)

%===================================================%
<p>
\frametitle{Time Series Analysis}
\begin{itemize}
\item Taking the logarithm of both sides of (16.5) results in a linear logarithmic trend equation,
log YT ¼ log b0 þ X log b1 (16:6)
\item The advantage of the transformation into logarithms is that the linear equation for trend analysis can be
applied to the logs of the values when the time series follows an exponential curve.
\item  The forecasted log values
for YT can then be reconverted to the original measurement units by taking the antilog of the values. We do not
demonstrate such analysis in this book.
\end{itemize}

%===================================================%
<p>
\frametitle{Time Series Analysis}
\begin{itemize}
\item Many time series for the sales of a particular product can be observed to include three stages: an
introductory stage of slow growth in sales, a middle stage of rapid sales increases, and a final stage of slow
growth as market saturation is reached.
\item For some products, such as structural steel, the complete set of three
stages may encompass many years. 
\item For other products, such as citizen band radios, the stage of saturation may
be reached relatively quickly. 
\item One particular trend curve that includes the three stages just described is the
S-shaped Gompertz curve, as portrayed in Fig. 16-2(b). 
\end{itemize}

%===================================================%
<p>
\frametitle{Time Series Analysis}
\textbf{Gompertz Trend}\\
An equation that can be used to fit the Gompertz trend
curve is
YT ¼ b0 b(b2) X
1 (16:7)
The values of b0, b1, and b2 are determined by first taking the logarithm of both sides of the equation, as
follows:
log YT ¼ log b0 þ ( log b1)bX
2 (16:8)


%===================================================%
<p>
\frametitle{Time Series Analysis}
Finally, the values to form the trend curve are calculated by taking the antilog of the values calculated
by Formula (16.8). The details of such calculations are included in specialized books on time series
analysis.


%===================================================%
<p>
\frametitle{Time Series Analysis}
\textbf{PART C : ANALYSIS OF CYCLICAL VARIATIONS}
Annual time series values represent the effects of only the trend and cyclical components, because the
seasonal and irregular components are defined as short-run influences. Therefore, for annual data the
cyclical component can be identified as being the component that would remain in the data after the influence of
the trend component is removed. This removal is accomplished by dividing each of the observed values by the
associated trend value, as follows:
Y
YT
¼
T   C
T
¼ C (16:9)

%===================================================%
<p>
\frametitle{Time Series Analysis}
\begin{itemize}
\item The ratio in (16.9) is multiplied by 100 so that the mean cyclical relative will be 100.0. A cyclical relative
of 100 would indicate the absence of any cyclical influence on the annual time series value. See Problem 16.2.
\item In order to aid in the interpretation of cyclical relatives, a cycle chart which portrays the cyclical relatives
according to year is often prepared. 
item The peaks and troughs associated with the cyclical component of the time
series can be made more apparent by the construction of such a chart. See Problem 16.3.
\end{itemize}

%===================================================%
<p>
\frametitle{Time Series Analysis}
\textbf{PART D  MEASUREMENT OF SEASONAL VARIATIONS}
\begin{itemize}
\item The influence of the seasonal component on time series values is identified by determining the seasonal
index number associated with each month (or quarter) of the year. 
\item The arithmetic mean of all 12 monthly index
numbers (or four quarterly index numbers) is 100. 
\item The identification of positive and negative seasonal
influences is important for production and inventory planning.
\end{itemize}

%===================================================%
<p>
\frametitle{Time Series Analysis}
EXAMPLE 2. 
\begin{itemize}
\item An index number of 110 associated with a given month indicates that the time series values for that month
have averaged 10 percent higher than for other months because of some positive seasonal factor. 
\item For instance, the unit sales
of men’s shavers might be 10 percent higher in June as compared with other months because of Father’s Day.
\item The procedure most frequently used to determine seasonal index numbers is the ratio-to-moving-average
method.
\item By this method, the ratio of each monthly value to the moving average centered at that month is first
determined. 
\end{itemize}

%===================================================%
<p>
\frametitle{Time Series Analysis}
Because a moving average based on monthly (or quarterly) data for an entire year would average
out the seasonal and irregular fluctuations, but not the longer-term trend and cyclical influences, the ratio of a
monthly (or quarterly) value to a moving average can be represented symbolically by
Y
Moving average
¼
T   C   S   I
T   C
¼ S   I (16:10)


%===================================================%
<p>
\frametitle{Time Series Analysis}
\begin{itemize}
\item The second step in the ratio-to-moving-average method is to average out the irregular component.
\item  This is
typically done by listing the several ratios applicable to the same month (or quarter) for the several years,
eliminating the highest and lowest values, and computing the mean of the remaining ratios.
\item  The resulting mean
is called a modified mean, because of the elimination of the two extreme values.
\item The final step in the ratio-to-moving-average method is to adjust the modified mean ratios by a
correction factor so that the sum of the 12 monthly ratios is 1,200 (or 400 for four quarterly ratios). See
Problem 16.4.
\end{itemize}

%===================================================%
<p>
\frametitle{Time Series Analysis}
\textbf{PART E: APPLYING SEASONAL ADJUSTMENTS}
\begin{itemize}

\item One frequent application of seasonal indexes is that of adjusting observed time series data by removing the
influence of the seasonal component from the data. 
\item Such adjusted data are called seasonally adjusted data, or
deseasonalized data. 
\item Seasonal adjustments are particularly relevant if we wish to compare data for different
months to determine if an increase (or decrease) relative to seasonal expectations has taken place.
\end{itemize}

%===================================================%
<p>
\frametitle{Time Series Analysis}
\textbf{EXAMPLE 3. }
\begin{itemize}
\item An increase in lawn fertilizer sales of 10 percent from April to May of a given year represents a relative
decrease if the seasonal index number for May is 20 percent above the index number for April. 
\item In other words, if an increase
occurs but is not as large as expected based on historical data, then relative to these expectations a relative decline in demand
has occurred.
\end{itemize}


%===================================================%
<p>
\frametitle{Time Series Analysis}
\begin{itemize}
\item The observed monthly (or quarterly) time series values are adjusted for seasonal influence by dividing each
value by the monthly (or quarterly) index for that month. 
\item The result is then multiplied by 100 to maintain the
decimal position of the original data.
\item The process of adjusting data for the influence of seasonal variations can
be represented by
\end{itemize}
%% EQUATION

Although the resulting values after the application of (16.11) are in the same measurement units as the
original data, they do not represent actual data. Rather, they are relative values and are meaningful for
comparative purposes only. See Problem 16.5.


%===================================================%
<p>
\frametitle{Time Series Analysis}
\textbf{PART F: FORECASTING BASED ON TREND AND SEASONAL FACTORS}
\begin{itemize}
\item A beginning point for long-term forecasting of annual values is provided by use of the trend line (16.2)
equation. 
\item However, a particularly important consideration in long-term forecasting is the cyclical component of
the time series. 
\item There is no standard method by which the cyclical component can be forecast based on
historical time series values alone, but certain economic indicators (see Section 16.7) are useful for anticipating
cyclical turning points.
\end{itemize}


%===================================================%
<p>
\frametitle{Time Series Analysis}
\begin{itemize}
\item For short-term forecasting, one possible approach is to deseasonalize the most-recent observed value and
then to multiply this deseasonalized value by the seasonal index for the forecast period. 
\item This approach assumes
that the only difference between the two periods will be the difference that is attributable to the seasonal
influence. 
\item An alternative approach is to use the projected trend value as the basis for the forecast and then adjust
it for the seasonal component. 
\item When the equation for the trend line is based on annual values, one must first
``step down" the equation so that is expressed in terms of months (or quarters). 
\end{itemize}

\end{document}
%===================================================%
<p>
\frametitle{Time Series Analysis}A trend equation based on annual
data is modified to obtain projected monthly values as follows:
YT ¼
b0
12
þ
b1
12  € X
12  € ¼
b0
12
þ
b1
144
X (16:12)
A trend equation based on annual data is modified to obtain projected quarterly values as follows:
YT ¼
b0
4
þ
b1
4  € X
4  € ¼
b0
4
þ
b1
16
X (16:13)


%===================================================%
<p>
\frametitle{Time Series Analysis}
\begin{itemize}
\item The basis for the above modifications is not obvious if one overlooks the fact that trend values are not
associated with points in time, but rather, with periods of time. 
item Because of this consideration, all three elements
in the equation for annual trend (b0, b1, and X) have to be stepped down.
\item By the transformation for monthly data in (16.12), the base point in the year that was formerly coded X ¼ 0
would be at the middle of the year, or July 1. 
\item Because it is necessary that the base point be at the middle of the
first month of the base year, or January 15, the intercept b0/12 in the modified equation is then reduced by 5.5
times the modified slope. A similar adjustment is made for quarterly data. 
\end{itemize}


%===================================================%
<p>
\frametitle{Time Series Analysis}
Thus, a trend equation which is
modified to obtain projected monthly values and with X ¼ 0 placed at January 15 of the base year is
YT ¼
b0
12
" (5:5)
b1
144  € þ
b1
144
X (16:14)

%===================================================%
<p>
\frametitle{Time Series Analysis}
Similarly, a trend equation that is modified to obtain projected quarterly values and with X ¼ 0 placed at the
middle of the first quarter of the base year is
YT ¼
b0
4
" (1:5)
bi
16  € þ
b1
16
X (16:15)
Problem 16.6 illustrates the process of stepping down a trend equation. After monthly (or quarterly) trend
values have been determined, each value can be multiplied by the appropriate seasonal index (and divided by
100 to preserve the decimal location in the values) to establish a beginning point for short-term forecasting.
See Problem 16.7.


%===================================================%
<p>
\frametitle{Time Series Analysis}
\textbf{PART G CYCLICAL FORECASTING AND BUSINESS INDICATORS}
\begin{itemize}
\item
As indicated in Section Part F, forecasting based on the trend and seasonal components of a time series is
considered only a beginning point in economic forecasting.
\item  One reason is the necessity to consider the likely
effect of the cyclical component during the forecast period, while the second is the importance of identifying the
specific causative factors that have influenced the time series variables.
\end{itemize}

%===================================================%
<p>
\frametitle{Time Series Analysis}
\begin{itemize}
\item For short-term forecasting, the effect of the cyclical component is often assumed to be the same as included
in recent time series values.
\item However, for longer periods, or even for short periods during economic instability,
the identification of the cyclical turning points for the national economy is important. 
\item Of course, the cyclical
variations associated with a particular product may or may not coincide with the general business cycle.
\end{itemize}

%===================================================%
<p>
\frametitle{Time Series Analysis}
\textbf{EXAMPLE 4.}
\begin{itemize} 
\item Historically, factory sales of passenger cars have coincided closely with the general business cycle for the
national economy.
\item On the other hand, sales of automobile repair parts have often been countercyclic with respect to the
general business cycle.
\end{itemize}

%===================================================%
<p>
\frametitle{Time Series Analysis}
\textbf{Leading Indicators}
\begin{itemize}
\item The National Bureau of Economic Research (NBER) has identified a number of published time series that
historically have been indicators of cyclic revivals and recessions with respect to the general business cycle.
\item One group, called leading indicators, has usually reached cyclical turning points prior to the corresponding
change in general economic activity. 
\item The leading indicators include such measures as average weekly hours
worked in manufacturing, value of new orders for consumer goods and materials, and a common stock price
index.
\end{itemize}

%===================================================%
<p>
\frametitle{Time Series Analysis}
\textbf{ Coincident Indicators}\\
\begin{itemize}
\item A second group, called coincident indicators, are time series which have generally had turning points
coinciding with the general business cycle.
\item Coincident indicators include such measures as the employment rate
and the index of industrial production. 
\end{itemize}

%===================================================%
<p>
\frametitle{Time Series Analysis}
\textbf{Lagging Indicators}\\
\begin{itemize}
\item The third group, called lagging indicators, are those time series for
which the peaks and troughs usually lag behind those of the general business cycle. 
\item Lagging indicators include
such measures as manufacturing and trade inventories and the average prime rate charged by banks.
\end{itemize}

%===================================================%
<p>
\frametitle{Time Series Analysis}
In addition to considering the effect of cyclical fluctuations and forecasting such fluctuations, specific
causative variables that have influenced the time series values historically should also be studied. Regression
and correlation analysis (see Chapters 14 and 15) are particularly applicable for such studies as the relationship
between pricing strategy and sales volume. Beyond the historical analyses, the possible implications of new
products and of changes in the marketing environment are also areas of required attention.

\end{document}
%===================================================%
<p>
\frametitle{Time Series Analysis}
16.8 FORECASTING BASED ON MOVING AVERAGES
A moving average is the average of the most recent n data values in a time series. This procedure can be
represented by
MA ¼
S (most recent n values)
n
(16:16)
As each new data value becomes available in a time series, the newest observation replaces the oldest
observation in the set of n values as the basis for determining the new average, and this is why it is called a
moving average.

%===================================================%
<p>
\frametitle{Time Series Analysis}
\begin{itemize}
\item The moving average can be used to forecast the data value for the next (forthcoming) period in the time
series, but not for periods that are farther in the future. 
\item It is an appropriate method of forecasting when there is
no trend, cyclical or seasonal influence on the data, which of course is an unlikely situation. 
\item The procedure
serves simply to average out the irregular component in the recent time series data. See Problem 16.8.
\end{itemize}

%===================================================%
<p>
\frametitle{Time Series Analysis}
\textbf{PART H  EXPONENTIAL SMOOTHING AS A FORECASTING METHOD}
Exponential smoothing is a method of forecasting that is also based on using a moving average, but it is a
weighted moving average rather than one in which the preceding data values are equally weighted. The basis for
the weights is exponential because the greatest weight is given to the data value for the period immediately
%CHAP. 16] TIME SERIES ANALYSIS AND BUSINESS FORECASTING 301
preceding the forecast period and the weights decrease exponentially for the data values of earlier periods.

%===================================================%
<p>
\frametitle{Time Series Analysis}

There are, in fact, several types of exponential smoothing models, as described in specialized books in business
forecasting. The method presented here is called simple exponential smoothing.
The following algebraic model serves to represent how the exponentially decreasing weights are
determined. Specifically, where a is a smoothing constant discussed below, the most recent value of the time
series is weighted by a, the next most recent value is weighted by a(1 2 a), the next value by a(1 2 a)2, and
so forth, and all the weighted values are then summed to determine the forecast:
^Y
Yt1 ¼ aYt þ a (1  a)Yt1 þ a (1  a)2Yt2 þ " " " þ a (1  a)kYtk (16:17)
where ^YYtþ1 ¼ forecast for the next period
a ¼ smoothing constant (0 # a # 1)
Yt ¼ actual value for the most recent period
Yt1 ¼ actual value for the period preceding the most recent period
Ytk ¼ actual value for k periods preceding the most recent period

%===================================================%
<p>
\frametitle{Time Series Analysis}

Although the above formula serves to present the rationale of exponential smoothing, its use is quite
cumbersome. A simplified procedure that requires an initial “seed” forecast but does not require the
determination of weights is generally used instead. The formula for determining the forecast by the simplified
method of exponential smoothing is
^Y
Ytþ1 ¼ ^YYt þ a (Yt  ^YYt) (16:18)
where ^YYtþ1 ¼ forecast for the next period
^Y
Yt ¼ forecast for the most recent period
a ¼ smoothing constant (0 # a # 1)
Yt ¼ actual value for the most recent period

%===================================================%
<p>
\frametitle{Time Series Analysis}
Because the most recent time series value must be available to determine a forecast for the following
period, simple exponential smoothing can be used only to forecast the value for the next period in the time
series, not for several periods into the future. The closer the value of the smoothing constant is set to 1.0, the
more heavily is the forecast weighted by the most recent results. See Problem 16.9

%===================================================%
<p>
\frametitle{Time Series Analysis}
\textbf{PAART J OTHER FORECASTING METHODS THAT INCORPORATE SMOOTHING}
\begin{itemize}
\item Whereas the moving average is appropriate as the basis for forecasting only when the irregular influence
causes the time series values to vary, simple exponential smoothing is most appropriate only when the cyclical
and irregular influences comprise the main effects on the observed values.
\item  In both methods, a forecast can be
obtained only for the next period in the time series, and not for periods farther into the future. 
\item Other
more complex methods of smoothing incorporate more influences and permit forecasting for several periods
into the future. These methods are briefly described below.
\item Full explanations and descriptions of these methods
are included in specialized textbooks on forecasting.
\end{itemize}

%===================================================%
<p>
\frametitle{Time Series Analysis}

Linear exponential smoothing utilizes a linear trend equation based on the time series data. However,
unlike the simple trend equation presented in Section 16.2, the values in the series are exponentially weighted
based on the use of a smoothing constant. As in simple exponential smoothing, the constant can vary from
0 to 1.0.

%===================================================%
<p>
\frametitle{Time Series Analysis}
Holt’s exponential smoothing utilizes a linear trend equation based on using two smoothing constants: one
to estimate the current level of the time series values and the other to estimate the slope.

%===================================================%
<p>
\frametitle{Time Series Analysis}
Winter’s exponential smoothing incorporates seasonal influences in the forecast. Three smoothing
constants are used: one to estimate the current level of the time series values, the second to estimate the slope of
the trend line, and the third to estimate the seasonal factor to be used as a multiplier.

%===================================================%
<p>
\frametitle{Time Series Analysis}
Autoregressive integrated moving average (ARIMA) models are a category of forecasting methods in which
previously observed values in the time series are used as independent variables in regression models. The most
widely used method in this category was developed by Box and Jenkins, and is generally called the Box-Jenkins
method. These methods make explicit use of the existence of autocorrelation in the time series, which is the
correlation between a variable, lagged one or more periods, with itself. As described in Section 15.8, the

%===================================================%
<p>
\frametitle{Time Series Analysis}
Durbin-Watson test serves to detect the existence of autocorrelated residuals (serial correlation) in time series
values. A value of the test statistic close to 2.0 supports the null hypothesis that no autocorrelation exists in the
time series. A value below 1.4 generally is indicative of strong positive serial correlation, while a value greater
than 2.6 indicates the existence of strong negative serial correlation.

%===================================================%
<p>
\frametitle{Time Series Analysis}

16.11 USING COMPUTER SOFTWARE
Specialized software is available in time series analysis and business forecasting, with much of it
incorporating the more sophisticated techniques that are described in the preceding section of this chapter.
Problems 16.10 through 16.13 illustrate the use of Execustat for determining a linear trend equation, doing
seasonal analysis, using the method of moving averages for the purpose of forecasting, and using simple
exponential smoothing for the purpose of forecasting.
Solved Problems

%===================================================%
<p>
\frametitle{Time Series Analysis}

TREND ANALYSIS
16.1 Table 16.1 presents sales data for an 11-year period for a graphics software company (fictional)
incorporated in 1990, as described in Example 1, and for which the time series data are portrayed by the
line graph in Fig. 16-1. Included also are worktable calculations needed to determine the equation for
the trend line. (a) Determine the linear trend equation for these data by the least-square method, coding
1990 as 0 and carrying all values to two places beyond the decimal point. Using this equation determine
the forecast of sales for the year 2001. (b) Enter the trend line on the line graph in Fig. 16.1.
ðaÞ YT ¼ b0 þ b1X
where XX ¼
SX
n
¼
55
11
¼ 5:00
Y
Y ¼
SY
n
¼
12:9
11
¼ 1:17
b1 ¼
SXY # n XX YY
SX 2 # nXX
2 ¼
85:2 # 11(5:00)(1:17)
385 # 11(5:00)2
¼
20:85
110
¼ 0:19
b0 ¼ YY # b1
X
X ¼ 1:17 # 0:19(5:00) ¼ 0:22
YT ¼ 0:22 þ 0:19X(with X ¼ 0 at 1990)
YT (2001) ¼ 0:22 þ 0:19(11) ¼ 2:31 (in millions)
CHAP. 16] TIME SERIES ANALYSIS AND BUSINESS FORECASTING 303
This equation can be used as a beginning point for forecasting, as described in Section 16.6. The slope of 0.19
indicates that during the 11-year existence of the company there has been an average increase in sales of 0.19
million dollars ($190,000) annually.
(b) Figure 16-3 repeats the line graph in Fig. 16-1, but with the trend line entered in the graph. The peak and
trough in the time series that were briefly discussed in Example 1 are now more clearly visible.

%===================================================%
<p>
\frametitle{Time Series Analysis}
ANALYSIS OF CYCLICAL VARIATIONS
16.2 Determine the cyclical component for each of the time series values reported in Table 16.1, utilizing the
trend equation determined in Problem 16.1.
Table 16.2 presents the determination of the cyclic relatives. As indicated in the last column of the table, each
cyclical relative is determined by multiplying the observed time series value by 100 and dividing by the trend value.
Thus, the cyclical relative of 90.0 for 1990 was determined by calculating 100(0.20)/0.22.
Table 16.1 Historical Sales for a Graphics Software
Firm, with Worktable to Determine the
Equation for the Trend Line
Year
Coded year
(X)
Sales, in
$millions (Y) X Y X 2
1990 0 $0.2 0 0
1991 1 0.4 0.4 1
1992 2 0.5 1.0 4
1993 3 0.9 2.7 9
1994 4 1.1 4.4 16
1995 5 1.5 7.5 25
1996 6 1.3 7.8 36
1997 7 1.1 7.7 49
1998 8 1.7 13.6 64
1999 9 1.9 17.1 81
2000 10 2.3 23.0 100
Total 55 $12.9 85.2 385
Fig. 16-3
% 304 TIME SERIES ANALYSIS AND BUSINESS FORECASTING [CHAP. 16

%===================================================%
<p>
\frametitle{Time Series Analysis}
16.3 Construct a cycle chart for the sales data reported in Table 16.1, based on the cyclical relatives
determined in Table 16.2.
The cycle chart is presented in Fig. 16-4, and includes the peak and trough observed previously in Fig. 16-3.

%===================================================%
<p>
\frametitle{Time Series Analysis}
MEASUREMENT OF SEASONAL VARIATIONS
16.4 Table 16.3 presents quarterly sales data for the graphics software company for which annual
data are reported in Table 16.1. Determine the seasonal indexes by the ratio-to-moving-average
method.
Table 16.4 is concerned with the first step in the ratio-to-moving-average method, that of computing the ratio
of each quarterly value to the 4-quarter moving average centered at that quarter.
The 4-quarter moving totals are centered between the quarters in the table because as a moving total of an even
number of quarters, the total would always fall between two quarters. For example, the first listed total of 1,500 (in
$1,000s) is the total sales amount for the first through the fourth quarters of 1995. Since four quarters are involved,
the total is centered vertically in the table between the second and third quarter.
Table 16.2 Determination of Cyclical Relatives
Coded year Sales, in $millions Cyclical relative
Year (X) Actual (Y) Expected (Y) 100 Y/Y
1990 0 0.20 0.22 90.9
1991 1 0.40 0.41 97.6
1992 2 0.50 0.60 83.3
1993 3 0.90 0.79 113.9
1994 4 1.10 0.98 112.2
1995 5 1.50 1.17 128.2
1996 6 1.30 1.36 95.6
1997 7 1.10 1.55 71.0
1998 8 1.70 1.74 97.7
1999 9 1.90 1.93 98.4
2000 10 2.30 2.12 108.5






FORECASTING BASED ON TREND AND SEASONAL FACTORS
16.6 Step down the trend equation developed in Problem 16.1 so that it is expressed in terms of quarters
rather than years. Also, adjust the equation so that the trend values are in thousands of dollars instead of
millions of dollars. Carry the final values to the first place beyond the decimal point.
\[ Y_T (quarterly) =  \frac{b_0}{4} -1.5 \left( \frac{b_1}{16} \right) + \left( \frac{b_1}{16} \right)X\]

\[  Y_T (quarterly) =  \frac{0.22}{4} -1.5 \left( \frac{0.19}{16} \right) + \left( \frac{0.19}{16} \right)X\]

\[ Y_T (quarterly) = 0.0550   0.0178 + 0.0119X
\[ Y_T (quarterly) = 0.0372 + 0.0119X\]


% http://people.su.se/~ma/R_intro/R_intro.pdf

Time-series

### Data in rows
Time-series data often appear in a form where series are in rows. As an example
I use data from the Swedish Consumer Surveys by the National Institute of
Economic Research containing three series: consumer condence index, a macro
index and a micro index.

First I saved the original data in a file in text-format. Before using the data as
input in R I used a text editor and kept only the values for the rst three series
separated with spaces. The series are in rows. The values of the three series are
listed in separate rows without variable names.

To read this file <macro.txt>, the following code puts the data in a matrix of
3 columns and 129 rows with help of <scan> and <matrix> before dening a
time-series object starting in 1993 with frequency 12 (monthly). The series are
named as <cci>,<macro.index> and <micro.index>. Notice that <matrix>
by default lls in data by columns.

\begin{verbatim}
> FILE <- "http://people.su.se/~ma/R_intro/macro.txt"
> macro <- ts(matrix(scan(FILE), 129, 3), start = 1993,
+ frequency = 12, names = c("cci", "macro.index",
+ "micro.index"))
\end{verbatim}
Here I give an example for creating lag values of a variable and adding it to a
time-series data set. See also <diff> for computing dierences.
Let us create a new time series data set, with the series in the data frame
<macro> adding lagged <cci> (lagged by 1 month). The function <ts.union>
puts together the series keeping all observations while <ts.intersect> would
keep only the overlapping part of the series.
> macro2 <- ts.union(macro, l.cci = lag(macro[,
+ 1], -1))
You can use the function aggregate to change the frequency of you time-series
data. The following example converts the frequency data. nfrequency=1 yields
annual data. FUN=mean computes the average of the variables over time. The
default is <sum>.
> aggregate(macro, nfrequency = 1, FUN = mean)
Time Series:
Start = 1993
End = 2002
Frequency = 1
cci macro.index micro.index
1993 -19.7583333 -33.3833333 -13.241667
1994 -0.2916667 14.1666667 -5.158333
