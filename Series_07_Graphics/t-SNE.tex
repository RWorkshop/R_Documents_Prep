Visualizing Data using t-SNE

%- http://www.jmlr.org/papers/v9/vandermaaten08a.html
Laurens van der Maaten, Geoffrey Hinton; 9(Nov):2579--2605, 2008.

Abstract

We present a new technique called "t-SNE" that visualizes high-dimensional data by giving each datapoint a location in a two or three-dimensional map. The technique is a variation of Stochastic Neighbor Embedding (Hinton and Roweis, 2002) that is much easier to optimize, and produces significantly better visualizations by reducing the tendency to crowd points together in the center of the map. t-SNE is better than existing techniques at creating a single map that reveals structure at many different scales. This is particularly important for high-dimensional data that lie on several different, but related, low-dimensional manifolds, such as images ofobjects from multiple classes seen from multiple viewpoints. For visualizing the structure of very large data sets, we show how t-SNE can use random walks on neighborhood graphs to allow the implicit structure of all of the data to influence the way in which a subset of the data is displayed. We illustrate the performance of t-SNE on a wide variety of data sets and compare it with many other non-parametric visualization techniques, including Sammon mapping, Isomap, and Locally Linear Embedding. The visualizations produced by t-SNE are significantly 
better than those produced by the other techniques on almost all of the data sets.

%
%%- http://scikit-learn.org/stable/modules/generated/sklearn.manifold.TSNE.html

t-distributed Stochastic Neighbor Embedding.
t-SNE [1] is a tool to visualize high-dimensional data. It converts similarities between data points to joint probabilities and tries to minimize the Kullback-Leibler divergence between the joint probabilities of the low-dimensional embedding and the high-dimensional data. t-SNE has a cost function that is not convex, i.e. with different initializations we can get different results.
It is highly recommended to use another dimensionality reduction method (e.g. PCA for dense data or TruncatedSVD for sparse data) to reduce the number of dimensions to a reasonable amount (e.g. 50) if the number of features is very high. This will suppress some noise and speed up the computation of pairwise distances between samples. For more tips see Laurens van der Maaten’s FAQ [2].

%=
