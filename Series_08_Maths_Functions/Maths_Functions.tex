
<p>
### {Managing Precision}

texttt{floor()} - 
texttt{ceiling()} - 
texttt{round()} - 
texttt{as.integer()} -


<pre>
<code>
> pi
[1] 3.141593
> floor(pi)
[1] 3
> ceiling(pi)
[1] 4
> round(pi,3)
[1] 3.142
> as.integer(pi)
[1] 3
</code>
</pre>
<p>

<p>
### {Basic Operations}
<p>
#### {Complex numbers}
<p>
#### {Trigonometric functions}
<p>
### {Matrices}

%end{document}



#### {exponentials, powers and logarithms}
<pre>
<code>
>x^y
>exp(x)
>log(x)
>log(y)
#determining the square root of x
>sqrt(x)
</code>
</pre>
<p>

<p>
#### {vectors}
<pre>
<code>
R handles vector objects quite easily and intuitively.

> x<-c(1,3,2,10,5)    #create a vector x with 5 components
> x
[1]  1  3  2 10  5
> y<-1:5#create a vector of consecutive integers
> y
[1] 1 2 3 4 5
> y+2   #scalar addition
[1] 3 4 5 6 7
> 2     y   #scalar multiplication
[1]  2  4  6  8 10
> y^2   #raise each component to the second power
[1]  1  4  9 16 25
> 2^y   #raise 2 to the first through fifth power
[1]  2  4  8 16 32
> y     #y itself has not been unchanged
[1] 1 2 3 4 5
> y<-y     2
> y     #it is now changed
[1]  2  4  6  8 10
</code>
</pre>
<p>


#### {Misc}
texttt{seq()} and texttt{rep()} are useful commands for constructing vectors with a certain pattern.

%end{document}

<p>
#### {Matrices}
A matrix refers to a numeric array of rows and columns.

One of the easiest ways to create a matrix is to combine vectors of equal
length using cbind(), meaning "column bind". Alternatively one can use rbind(), meaning ``row bind".



#### {Matrices Inversion}

#### {Matrices Multiplication}


<p>
#### {Data frame}
A Data frame is

