#### {Useful Statistical Commands}

* ``mean()`` mean of a data set
* ``median()`` median of a data set
* ``length()`` Sample Size
* ``IQR()`` Inter-Quartile Range of a sample
* ``var()`` Variance of a sample
* ``sd()`` Standard Deviation  of a sample
* ``range()`` Range of a data set
* ``fivenum()`` Tukey's five number summary



\subsubsection{exponentials, powers and logarithms}
<pre>
<code>
>x^y
>exp(x)
>log(x)
>log(y)
#determining the square root of x
>sqrt(x)
</code>
</pre>

\subsection{vectors}
<pre>
<code>
R handles vector objects quite easily and intuitively.

> x<-c(1,3,2,10,5)    #create a vector x with 5 components
> x
[1]  1  3  2 10  5
> y<-1:5              #create a vector of consecutive integers
> y
[1] 1 2 3 4 5
> y+2                 #scalar addition
[1] 3 4 5 6 7
> 2*y                 #scalar multiplication
[1]  2  4  6  8 10
> y^2                 #raise each component to the second power
[1]  1  4  9 16 25
> 2^y                 #raise 2 to the first through fifth power
[1]  2  4  8 16 32
> y                   #y itself has not been unchanged
[1] 1 2 3 4 5
> y<-y*2
> y                   #it is now changed
[1]  2  4  6  8 10
</code>
</pre>

\subsubsection{Misc}
``seq()`` and ``rep()`` are useful commands for constructing vectors with a certain pattern.

%\end{document}

\subsection{Matrices}
A matrix refers to a numeric array of rows and columns.

One of the easiest ways to create a matrix is to combine vectors of equal
length using cbind(), meaning "column bind". Alternatively one can use rbind(), meaning ``row bind".


\subsubsection{Matrices Inversion}
\subsubsection{Matrices Multiplication}


\subsection{Data frame}
A Data frame is
\newpage

%------------------------------------------------------------------------------------------------%

\chapter{Descriptive Statistics}

\section{Basic Statistics}


<pre>
<code>
> X=c(1,4,5,7,8,9,5,8,9)
> mean(X);median(X)       #mean and median of vector
[1] 6.222222
[2] 7
> sd(X)                   #standard deviation of Vector
[1] 2.682246
> length(X)               #sample size of vector
[1] 9
> sum(X)
[1] 56
> X^2
[1]  1 16 25 49 64 81 25 64 81
> rev(X)
[1] 9 8 5 9 8 7 5 4 1
> sort(X)                 #items in ascending order
[1] 1 4 5 5 7 8 8 9 9
> X[1:5]
[1] 1 4 5 7 8
</code>
</pre>



\section{Summary Statistics}
The ``R} command ``summary()`` returns a summary statistics for a simple dataset.
The ``R} command ``fivenum()`` returns a summary statistics for a simple dataset, but without the mean.
Also, the quartiles are computed a different way.


<pre>
<code>
> summary(mtcars$mpg)
Min. 1st Qu.  Median    Mean 3rd Qu.    Max.
10.40   15.43   19.20   20.09   22.80   33.90 
>
> fivenum(mtcars$mpg)
[1] 10.40 15.35 19.20 22.80 33.90
</code>
</pre>





\section{Bivariate Data}
 <pre>
<code>
> Y=mtcars$mpg
> X=mtcars$wt
>
> cor(X,Y)          #Correlation
[1] -0.8676594
>
> cov(X,Y)          #Covariance
[1] -5.116685
</code>
</pre>



#### {Set Theory Operations}

* ``union()`` union of sets A and B
* ``intersect()`` intersection of sets A and B
* ``setdiff()`` set difference A-B (order is important)


<pre>
<code>
x = 5:10
y = 8:12
union(x,y)
intersect(x,y)
setdiff(x,y)
setdiff(y,x)
<\code>
</pre>
%=============================================================================%


% R Class : Set Theory and Sampling

%# Sampling
%# - Sampling With Replacement
%# - Sampling Without Replacement

%=============================================================================%
#### {Set Theory with ``R} }


* Union
* Intersection
* Set Diffference

<pre>
<code>
X = 5:10
Y = 8:12
<\code>
</pre>

<pre>
<code>

union(X,Y)
# [1]  5  6  7  8  9 10 11 12
intersect(X,Y)
# [1]  8  9 10

<\code>
</pre>
%=============================================================================%


\section{R Class : Set Theory and Sampling}

%=============================================================================%
%# Sampling
%# - Sampling With Replacement
%# - Sampling Without Replacement

### Set Theory with ``R``


* Union
* Intersection
* Set Diffference

<pre>
<code>
X = 5:10
Y = 8:12
<\code>
</pre>
\end{frame}
%=============================================================================%

\frametitle{Set Theory with ``R} }
<pre>
<code>

union(X,Y)
# [1]  5  6  7  8  9 10 11 12
intersect(X,Y)
# [1]  8  9 10

<\code>
</pre>
\end{frame}
%=============================================================================%
\end{document}

#### {Set Theory Operations}

* ``union()`` union of sets A and B
* ``intersect()`` intersection of sets A and B
* ``setdiff()`` set difference A-B (order is important)


<pre>
<code>
x = 5:10
y = 8:12
union(x,y)
intersect(x,y)
setdiff(x,y)
setdiff(y,x)
<\code>
</pre>
%---------------------------------------------------------%
#### {The Birthday function}
The R command ``pbirthday()`` computes the probability of a coincidence of a number of randomly chosen people sharing a birthday, given that there are n people to choose from.
Suppose there are four people in a room. The probability of two of them sharing a birthday is computed as about 1.6 \%
<code>
> pbirthday(4)
[1] 0.01635591
<\code>

How many people do you need for a greater than 50\% chance of a shared birthday? (choose from 23,43,63,83)?
%---------------------------%



#### {Other Mathematical Functions}
%-----------------------------------------------------------------------%
Complex numbers
<pre>
<code>
x = -1 ;  sqrt(x)  ;  str(x) ; # variable is defined as numeric, not complex.
y = -1 +0i ;  sqrt(y)  ;  str(y) ;    #variable is defined as complex .

<\code>
</pre>
%-----------------------------------------------------------------------%
Trigonometric  Functions

<pre>
<code>
pi#returns the value of pi to six decimal places
sin(3.5*pi)# correct answer is -1
cos(3.5*pi)# correct answer is zero

<\code>
</pre>
%-----------------------------------------------------------------------%

%============================================================================================================== %
#### {Generating Random Numbers}
R is very useful for performing simulations.

<pre>
<code>
#generate a random number between 0 and 1
runif(1)
#generate four random numbers between 0 and 6 
runif(4,min=0,max=6)   
<\code>
</pre>
Random numbers can be discretized using the "floor()" or "Ceiling()" functions. Suppose we wish to simulate four throws of a dice.

X = ceiling (runif(4,min=0,max=6))
Y = floor (runif(4,min=1,max=7))
X+Y

%================================================================================================================= %
#### {Section 2: Basic Mathematical operations}
Trigonometric and power functions

Integration

<pre>
<code>
integrate(sin, lower =0, upper = 3)
integrate(dnorm, -1.96, 1.96) # standard normal distribution
integrate(dnorm, 0, Inf)   # standard normal distribution
<\code>
</pre>
Complex numbers 

``R} has a small number of built-in constants.
\begin{description}
\item[``LETTERS``]: the 26 upper-case letters of the Roman alphabet;

\item[``letters``]: the 26 lower-case letters of the Roman alphabet;

\item[``month.abb``]: the three-letter abbreviations for the English month names;

\item[``month.name``]: the English names for the months of the year;

\item[``pi``]: the ratio of the circumference of a circle to its diameter.

