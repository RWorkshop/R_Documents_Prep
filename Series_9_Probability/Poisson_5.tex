\documentclass[a4]{beamer}
\usepackage{amssymb}
\usepackage{graphicx}
\usepackage{subfigure}
\usepackage{newlfont}
\usepackage{amsmath,amsthm,amsfonts}
%\usepackage{beamerthemesplit}
\usepackage{pgf,pgfarrows,pgfnodes,pgfautomata,pgfheaps,pgfshade}
\usepackage{mathptmx}  % Font Family
\usepackage{helvet}   % Font Family
\usepackage{color}


\setbeamercovered{dynamic}

\title[MA4704]{Technology Mathematics 4 (Statistics) \\ {\normalsize MA4704 Lecture 3C}}
\author[Kevin O'Brien]{Kevin O'Brien \\ {\scriptsize Kevin.obrien@ul.ie}}
\date{Spring Semester 2013}
\institute[Maths \& Stats]{Dept. of Mathematics \& Statistics, \\ University \textit{of} Limerick}

\renewcommand{\arraystretch}{1.5}

\begin{document}

\begin{frame}
\titlepage
\end{frame}

%---------------------------------------------------------------------------%
\frame{
\frametitle{Characteristics of a Poisson Experiment}
A Poisson experiment is a statistical experiment that has the following properties:
\begin{itemize}
\item The experiment results in outcomes that can be classified as successes or failures.
\item The average number of successes (m) that occurs in a specified region is known.
\item The probability that a success will occur is proportional to the size of the \textbf{\emph{region}}.
\item The probability that a success will occur in an extremely small region is virtually zero.
\item The \texttt{pois} family of functions are used to compute probabilities and quantiles.
\end{itemize}
Note that the specified region could take many forms. For instance, it could be a length, an area, a volume, a period of time, etc.
}


%---------------------------------------------------------------------------%
\frame{
\frametitle{The Poisson Probability Distribution}
\begin{itemize}
\item A Poisson random variable is the number of successes that result from a Poisson experiment.
\item The probability distribution of a Poisson random variable is called a Poisson distribution.
\item This distribution describes the number of occurrences in a unit period (or space)
\item The expected number of occurrences is $m$.
\item \text{R} refers to the mean number of occurrences as \texttt{lambda} rather than \texttt{m}. 
\end{itemize}
}

%---------------------------------------------------------------------------%
\frame{
\frametitle{Poisson Formulae}
The probability that there will be $k$ occurrences in a unit time period is denoted $P(X=k)$, and is computed as below. Remark: This is known as the probability density function. The corresponding \texttt{R} command is \texttt{dpois()}.
\Large
\[ P(X = k)=\frac{m^k e^{-m}}{k!} \]


}
%---------------------------------------------------------------------------%
\frame{
	\frametitle{The Poisson Probability Distribution}
	\begin{itemize}
		\item The number of occurrences in a unit period (or space)
		\item The expected number of occurrences is $m$
		\item Given the mean number of successes ($m$) that occur in a specified region, we can compute the Poisson probability based on the following formula.
	\end{itemize}
}

%---------------------------------------------------------------------------%
\frame{
\frametitle{Poisson Formulae}
Given that there is on average 2 occurrences per hour, what is the probability of no occurrences in the next hour? \\ i.e. Compute $P(X=0)$ given that $m=2$
\Large
\[ P(X = 0)=\frac{2^0 e^{-2}}{0!} \]
\normalsize
\begin{itemize}
\item $2^0$ = 1
\item $0!$ = 1
\end{itemize}
The equation reduces to
\[ P(X = 0)=e^{-2} = 0.1353\]
}
%---------------------------------------------------------------------------%
\frame{
\frametitle{Poisson Formulae}
What is the probability of one occurrences in the next hour? \\ i.e. Compute $P(X=1)$ given that $m=2$
\Large
\[ P(X = 1)=\frac{2^1 e^{-2}}{1!} \]
\normalsize
\begin{itemize}
\item $2^1$ = 2
\item $1!$ = 1
\end{itemize}
The equation reduces to
\[ P(X = 1) = 2 \times e^{-2} = 0.2706\]
}
%---------------------------------------------------------------%
\frame{
\frametitle{Poisson Distribution (Example)}
\begin{itemize}

\item Suppose that electricity power failures occur according to a Poisson distribution
with an average of 2 outages every twenty weeks. \item Calculate the probability that there will
not be more than one power outage during a particular week.
\end{itemize}

\textbf{Solution:}

\begin{itemize}
\item The average number of failures per week is: $m = 2/20 = 0.10$

\item ``Not more than one  power outage" means we need to compute and add the probabilities for ``0 outages" plus ``1 outage".
\end{itemize}

}

%---------------------------------------------------------------%
\frame{
\frametitle{Poisson Distribution (Example)}

Recall: \[P(X = k) = e^{-m}\frac{m^k}{k!}\]


\begin{itemize}

\item $P(X = 0)$

\[P(X = 0) = e^{-0.10}\frac{0.10^0}{0!} = e^{-0.10} = 0.9048\]


\item $P(X = 1)$

\[P(X = 1) = e^{-0.10}\frac{0.10^1}{1!} = e^{-0.10}\times 0.1 = 0.0905\]

\item $P(X \leq 1)$

\[P(X \leq 1) = P(X = 0) + P(X = 1) = 0.9048 + 0.0905 = 0.995\]

\end{itemize}
}

\frame{
\frametitle{Implementation using \texttt{R}}

\begin{itemize}
\item Probability Density Function $P(X = k)$
\begin{itemize}
\item For a given poisson mean $m$, which in \texttt{R} is specified as \texttt{lambda} 
\item \texttt{dpois(k,lambda = ...)} 
\end{itemize}
\item Cumulative Density Function $P(X \leq k)$
\begin{itemize}
\item \texttt{ppois(k,lambda = ...)}
\end{itemize}
\end{itemize}

}


\begin{frame}[fragile]
\frametitle{Implementation using \texttt{R}}
From before: $P(X = 0)$ given than the mean number of occurrences is 2.

\begin{verbatim}
> dpois(0,lambda=2)
[1] 0.1353353
> dpois(1,lambda=2)
[1] 0.2706706
> dpois(2,lambda=2)
[1] 0.2706706
\end{verbatim}

\end{frame}


\begin{frame}[fragile]
\frametitle{Implementation using \texttt{R}}
Compute the cumulative distribution functions for the values $k=\{0,1,2\}$, given that the mean number of occurrences is 2

\begin{verbatim}
> ppois(0,lambda=2)
[1] 0.1353353
> ppois(1,lambda=2)
[1] 0.4060058
> ppois(2,lambda=2)
[1] 0.6766764
\end{verbatim}

\end{frame}


\end{document}


\end{document}
