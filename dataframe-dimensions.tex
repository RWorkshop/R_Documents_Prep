

%---------------------------------------------------------------------------------------------%
<p>
####        * {Dimensions of a data set}
We have remarked that some data sets are very large. This is perhaps a good place to consider summary information about data objects.
For a simple vector – a useful command to determine the length (remark: sample size) is the function texttt{length()}.
<pre>
<code>
> Y=5:9
> length(Y)
[1] 5
</code>
</pre>
<p>
%---------------------------------------------------------------------------------------------%
For more complex data sets (and data frames, which we will see later) – we have several tools for assessing the size of data.
<pre>
<code>
> dim(iris)  # dimensions of data set
[1] 150   5
> nrow(iris) # number of rows
[1] 150
> ncol(iris) # number of columns
[1] 5
</code>
</pre>
<p>
%---------------------------------------------------------------------------------------------%
We can also determine the row names and column names using the functions texttt{rownames()} and texttt{colnames()}.
If there are no specific row or column names – the command will just return the indices.

<pre>
<code>
> colnames(iris)
[1] "Sepal.Length" "Sepal.Width"  "Petal.Length" "Petal.Width"  "Species"

</code>
</pre>
