\documentclass{article}
\usepackage{natbib}
\usepackage{vmargin}
\usepackage{graphicx}
\usepackage{epsfig}
\usepackage{subfigure}
%\usepackage{amscd}
\usepackage{verbatim}   % useful for program listings
\usepackage{color}      % use if color is used in text
\usepackage{hyperref}   % use for hypertext links, including those to external documents and URLs
\usepackage{amssymb}
\usepackage{amsbsy}
\usepackage{amsthm, amsmath}
%\usepackage[dvips]{graphicx}
\bibliographystyle{chicago}
\renewcommand{\baselinestretch}{1.1}


\begin{document}


\section*{Exercise 22}

1. The following are measurements (in mm) of a critical
dimension on a sample of engine crankshafts:

  \begin{array}{|cccc|}

    224.120 & 224.017 & 223.976 & 223.961 \\
    224.089 & 223.982 & 223.980 & 223.989  \\
    223.960 & 223.902  & 223.987 & 224.001  \\

  \end{array}


(a) Calculate the mean and standard deviation for these data.\\
(b) The process mean is supposed to be $\mu$ = 224mm. Is this the
case? Give reasons for your answer.\\
(c) Construct a 99\% confidence interval for these data and
interpret.\\
(d) Check that the normality assumption is valid using 2 suitable
plots.

\end{document}
\newpage

\textbf{Assignment 5}\\
15\% of overall mark\\
Thursday 22nd April 3pm\\
Short questions on confidence intervals, test statistics, p-values ,etc\\
Pen and paper exam\\
45 minutes long\\
Open book\\\\

\textbf{Assignment 6}\\
20\% of overall mark\\
Project Work\\
Each student will be give some data sets.\\
The objective of the assignment is to utilize course material for these data sets\\
Submission by PDF document\\
Deadline: Thursday 29th April 6pm
\newpage
%--------------------------------------------------------------%
\begin{itemize}
\item Exercise: Multiple plots in a single graphic window.
\item Exercise: Adding Legends to plots.

\item Exercise: Linear Regression models.
\item Exercise: Correlation test.
\item Exercise: Shapiro Wilk test.
\item Exercise: Paired $t-$ test.
\end{itemize}

%--------------------------------------------------------------%

Forthcoming material
\begin{itemize}
\item plots (Continuing from before easter)
\begin{itemize}
  \item Par() command to set graphical parameters.
  \item Box plots
\end{itemize}
\item Linear Regression (lm()).
\item Confidence intervals
\item Hypothesis testing(p-values)
\item Inference on Linear regression
\item Special tests (Shapiro - Wilk test, correlation test)
\item outliers
\item Analysis of variance (ANOVA)
\end{itemize}
\newpage



%--------------------------------------------------------------%
\begin{itemize}
\item Week 10  One class - Thursday 8th April
\item Week 11  Two classes on Monday 12th April
\begin{itemize}
\item deadline for Assignment 4a Monday 8pm
\item email kevin.obrien@ul.ie
\end{itemize}
\item Week 11  One class on Thursday 15th April
\item Week 12  Two classes on Monday 19th April
\item Week 12  One class on Thursday 22nd April
\item Week 13  Assignment 5 on Monday 26th April
\begin{itemize}
\item Short questions
\item material from Week 11 onwards ( last 7 classes)
\item open book exam
\item worth 15%
\end{itemize}
\item Week 13  one class on Monday 26th April
\item Week 13  one class on Thursday 28th April
\item Week 14  deadline for Assignment 6
\end{itemize}
%--------------------------------------------------------------%

\newpage
\




1) Create a data frame for the data ( see `Car.R')
\begin{verbatim}
> Car<-data.frame(Make,Model,Cylinder,Weight,Mileage,Type)
\end{verbatim}

2) Find the `names' and the dimensions of the Car data frame.
\begin{verbatim}
> names(Car)
[1] "Make" "Model" "Cylinder" "Weight" "Mileage" "Type"
> dim(Car)
[1] 10  6
\end{verbatim}

3) Find the mean and standard deviation of the weight of the cars.
\begin{verbatim}
> mean(Car$Weight)
[1] 2858
> sd(Car$Weight)
[1] 543.7943
\end{verbatim}

4) Subset the data so that only cars above 2500 kgs are considered. (Call the subset 'cars2500')
\begin{verbatim}
cars2500 = Car[Car$Weight>2500,]
\end{verbatim}

5) Make a subset of the small cars (call the subset `smallcar').
\begin{verbatim}
> smallcar=Car[Car$Type=="Small",]
> smallcar
        Make  Model Cylinder Weight Mileage  Type
3       Ford Escort       V4   2345      33 Small
4      Eagle Summit       V4   2560      33 Small
5 Volkswagen  Jetta       V4   2330      26 Small
>
\end{verbatim}

6) What is the correlation coefficient between weight and mileage?

\begin{verbatim}
> cor(Car$Weight,Car$Mileage)
[1] -0.8759577
\end{verbatim}

7) Make a plot of weights against mileage (include a title).

\begin{verbatim}
>plot(Car$Weight, Car$Mileage, main="Weight vs. Mileage")
\end{verbatim}

Alternatively
\begin{verbatim}
>attach(Car)
>plot(Weight, Mileage, main="Weight vs. Mileage")
\end{verbatim}

\newpage


\newpage

\begin{verbatim}
# Car.R
> Make<-c("Honda","Chevrolet","Ford","Eagle","Volkswagen",
+"Buick","Mitsbusihi", "Dodge","Chrysler","Acura")
>
> Model<-c("Civic","Beretta","Escort","Summit","Jetta",
+"Le Sabre","Galant", "Grand Caravan","New Yorker","Legend")
>
> rep("V6",3)               # repeat three times
[1] "V6" "V6" "V6"
> Cylinder<-c(rep("V4",5),"V6","V4",rep("V6",3))
> Cylinder
[1] "V4" "V4" "V4" "V4" "V4" "V6" "V4" "V6" "V6" "V6"
> Weight<-c(2170,2655,2345,2560,2330,3325,2745,3735,3450,3265)
> Mileage<-c(33,26,33,33,26,23,25,18,22,20)
> Type<-c("Sporty","Compact",rep("Small",3),"Large",
+"Compact","Van",rep("Medium",2))
\end{verbatim}

\end{document}
