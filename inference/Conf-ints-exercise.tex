
Task 1: Calculate the 95\% confidence limits for the true pH
utilizing $R$.


Solution. We are using Student t distribution with six degrees of
freedom and the following code gives us the confidence interval
for this problem.
\begin{verbatim}
>x <- c(5.12, 5.20, 5.15, 5.17, 5.16, 5.19, 5.15)
>n =length(x)
>alpha =0.05
>stderr =sd(x)/sqrt(n)
>LB=mean(x)+qt(alpha/2,6)* stderr
>UB=mean(x)+qt(1-alpha/2,6)* stderr
>LB
#[1] 5.137975
>UB
#[1]5.187739
\end{verbatim}


\subsubsection{example}
A survey of study habits wishes to determine whether the mean
study hours completed by women at a particular college is higher
than for men at the same college. A sample of $n_1$ = 10 women and
$n_2$ = 12 men were taken, with mean hours of study $\bar{x}_1$ =
120 and $\bar{x}_2$ = 105 respectively. The standard deviations
were known to be $\sigma_1$ = 28 and $\sigma_2$ = 35.

The hypothesis being tested is:

\begin{eqnarray}
H_{0}: \mu_1 = \mu_2\qquad \qquad (\mu_1 - \mu_2= 0)\\
H_{a}: \mu_1 \neq \mu_2 \qquad \qquad (\mu_1 - \mu_2 \neq 0)
\end{eqnarray}

In $R$, the test statistic is calculated using:
\end{frame}
%%%%%%%%%%%%%%%%%%%%%%%%%%%%%%%%%%%%%%%%%%%%%%%%%%%%%%%%%%%%%%%%%%%%%%%%%%%%%%%
\begin{framed}[fragile]

\begin{verbatim}
xbar1 <- 120
xbar2 <- 105
sd1 <- 28
sd2 <- 35
n1 <- 10
n2 <-12
TS <- ( (xbar1 - xbar2) - (0) )/sqrt( (sd1^2/n1) + (sd2^2/n2) )
TS
[1] 1.116536
\end{verbatim}
Now need to calculate the critical value or the p-value.
\end{frame}
%%%%%%%%%%%%%%%%%%%%%%%%%%%%%%%%%%%%%%%%%%%%%%%%%%%%%%%%%%%%%%%%%%%%%%%%%%%%%%%
\begin{framed}[fragile]

The critical value can be looked up using qnorm. Since this is a
one-tailed test and there is a > sign in $H_1$:

\begin{verbatim}
qnorm(0.95)
[1] 1.644854
\end{verbatim}

Since the test statistic is less than the critical value ( 1.116536 < 1:645 )there is not enough evidence to reject $H_0$
and conclude that the population mean hours study for women is
not higher than the population mean hours study for men.

\end{frame}
%%%%%%%%%%%%%%%%%%%%%%%%%%%%%%%%%%%%%%%%%%%%%%%%%%%%%%%%%%%%%%%%%%%%%%%%%%%%%%%
\begin{framed}[fragile]

The p-value is determined using pnorm.

Careful! Remember pnorm
gives the probability of getting a value LESS than the value specified. We want the probability of getting a value greater than
the test statistic.
\begin{verbatim}
1-pnorm(1.116536) # OR pnorm(1.116536, lower.tail=FALSE)
[1] 0.1320964
\end{verbatim}
\newpage

