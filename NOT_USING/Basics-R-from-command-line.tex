%=====================================================================================================================%
\begin{frame}
### Using R at the command line 

\begin{itemize}
\item R is most easily used in an interactive manner, typing code into the command line and R gives you an response. 

\item Questions are asked and answered on the command line. To start up R's command line you can do the following: in Windows find the R icon and double click. 

\item Other operating systems may have different ways.

\item R can be started in the usual way by double-clicking on the R icon on the desktop.
\end{itemize}

\begin{itemize}
\item  The > is called the prompt, used to indicate where you are to type. If a command is too long to fit on a line, a + is used for the continuation prompt.
\item If a command is not complete at the end of a line, R will give the "+" prompt on second and subsequent lines and continue to read input until the command is syntactically complete.
\item Commands are separated either by a semi-colon (;), or by a newline. Elementary commands can be grouped together into one compound expression by braces ({ and }). 
\end{itemize}


Comments can be put almost anywhere, starting with a hashmark (#); everything after "#" is a comment.
\end{frame}
%=====================================================================================================================%
\begin{frame}
The R console opens with information and then a prompt mark  >  it is ready to accept commands.

\frametitle{The Assignment operator}

\begin{itemize}
\item The assignment operator is a "=". \item This is valid as of R version 1.4.0.
\item  Previously it was (and still can be) a "<-".
\item 
Both will be used, although, you should learn one and stick with it.
\end{itemize}

