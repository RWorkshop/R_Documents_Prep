%=====================================================================================================================%
\begin{frame}

Using R's help commands

R has an inbuilt help facility. To get more information on any specific named function, for example “boxplot”, the command is: ?boxplot


?boxplot		# access help on boxplots
help(Im)        # access help on "Im"



On most R installations help is available in HTML format by running help.start() which will launch a Web browser that allows the help pages to be browsed with hyperlinks. 
 



help.start()



Getting help in R
R has a built-in help facility. To get more information on any specific function, e.g. sqrt(), the command is
> help(sqrt)
An alternative is
> ? sqrt
We can also obtain help on features specified by special characters.
The argument must be enclosed in single or double quotes (e.g. "[[")
> 
Help is also available in HTML format by running
> help.start()


\end{frame}
%=====================================================================================================================%
\begin{frame}
help (sqrt)   	      # create a vector x
?exp  	   		  # edit the values using spreadsheet interface.
help("[[")	         # print to screen

<p>


R was originally designed as a command language.  
Commands were typed into a text-based input area on the computer screen and the program responded with a response to each command.

R is an open source software package, meaning that the code written to implement the various functions can be freely examined and modified.
R can be installed free of charge from the R-project website.

An online manual in “An introduction to R” is available via the R help system. 
Type “help.start ()” at the command prompt to access it, R has many features in common with functional and object oriented programming languages.



In particular functions in R are treated as objects that can be manipulated or used recursively, example TSA book3

In common with functional languages, assignments in R can be avoided, but they are useful for clarity and convenience. 
In addition R runs faster when loops are avoided, which can often be achieved using matrix calculation instead however, thus results in obscure looking code.

<p>

Functions in R can be treated as "objects" that can be manipulated or used recursively.

R shares many aspects with both Object orietated and functional programming languages. 
All data in R is stored an objects, which have a range of "methods" available. 
The "class" of an object can be found using the class() function.

Using the Help functions

<p>

Embedded help commands "help()" and "help.search()" are good starting points to gather information.
Note that "help.search()" opens a web browser linked to the local manual pages.
 

R works best if you have a dedicated folder for each separateproject - called the working folder.Create the directory/folder that will be used as the working folder, e.g. create a folder on your desktop titled Your_name by right-clicking, then clicking New > Folder. 
 
Right-click on an existing R icon and click Copy.   
In the working folder, right-click and click Paste. The R icon will appear in the folder. As with many advanced programming languages, R distinguishes between several types of object. Those types includes scalar, vector, matrix, time, series data frames functions and graphics. The R function str applied to any R onject, including R functions.To start R in Windows, double click the R icon. To start R in Unix or Linux, type ‘R’ at the command prompt. 

