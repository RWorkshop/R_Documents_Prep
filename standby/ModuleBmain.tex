##  Inference procedures

In this section, we will look at various inference procedures ( confidence intervals and hypothesis tests that R can implement. We have already encountered the correlation test. 
As with every hypothesis test, there is a null hypothesis and an alternative hypothesis.
For each procedure, these will be formulated separately. The outcome of the test is either the rejection or the failure to reject the null hypothesis.

The key characteristics of the output of each test is the p-value. If the p-value is below a certain threshold ( we will use 1% or 0.01) we reject the null hypothesis, and accept the alternative hypothesis. Otherwise we fail to reject the null hypothesis.

It is also possible to implement one tailed tests (“greater than” or “less than tests”) by specifying an additional argument. You can use the alternative="less" or alternative="greater" option to specify a one tailed test. Access the help command for instructions on how to do this. 

For the sake of time we will solely use the two tailed test (tests of equality) for this 
module.

\subsection{Confidence intervals}
With most of these procedures, a confidence interval is specified automatically. We will interpret each confidence interval on a case by case basis.




