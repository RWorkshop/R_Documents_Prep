%=====================================================================================================================%
\begin{frame}[fragile]
\frametitle{Creating Dat}
Defining Variables 
R is case sensitive.
A convention is to use define a variable name with a capital letter. This reduces the chance of overwriting inbuild R functions, which are usually written in lowercase letters.
\end{frame}
%=====================================================================================================================%
\begin{frame}[fragile]
Commonly used variable names such as x,y and z (in lower case letters) are not "reserved".

x = 2           # create variable x and assign the value 2
y <- 4          # create variable y and assign the value 4
\end{frame}
%=====================================================================================================================%
\begin{frame}[fragile]
\begin{framed}
\begin{verbatim}
5 -> z          # create variable z and assign the value 5

x  #print x to screen
y  #print y to screen
z  #print z to screen
\end{verbatim}

\end{framed}

Remark: The value of each variable is prefaced with a " [1] ". This indicates that the value is a vector. More on that later.

\end{frame}

%=====================================================================================================================%
\begin{frame}[fragile]
\frametitle{Basic Mathematical Calculations with R}

We will briefly look at how R accomplished basic calculations.


\begin{framed}
\begin{verbatim}

x*y			# multiplication
x/z			# division

x^2			# powers
sqrt(x)		# square root

exp(z)		 # exponentials   
log(y)		 # logarithms

pi             # returns the value of pi to six decimal places
\end{verbatim}
\end{framed}
 
Complex numbers , Trigonometric  Functions and Binomial Coefficients

\end{frame}
%=====================================================================================================================% 
\begin{frame}
\frametitle{R as a calculator}
Binomial coefficients are computed using the choose() command.

\begin{framed}
\begin{verbatim}

J = -1 ;  sqrt(J)  ;  str(J) ;      # variable is defined as numeric, not complex.
K = -1 +0i ;\begin{frame}  sqrt(K)  ;  str(K) ;  # variable is defined as complex .


sin(3.5*pi)             # correct answer is -1
cos(3.5*pi)             # correct answer is zero
 
choose(6,2)             # From 6 how many ways of choosing items.
\end{verbatim}
\end{framed} 
\end{frame}
%=====================================================================================================================% 
\begin{frame}
\frametitle{R as a calculator}
Data Vectors

x=c(1,3,4,5,6,7)
y=c("R","G","B")

\end{frame}
