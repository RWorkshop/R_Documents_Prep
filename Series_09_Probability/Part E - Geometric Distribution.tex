\documentclass{beamer}

\usepackage{amsmath}

\begin{document}
\begin{frame}

\end{frame}
%-------------------------------------------------------------%
\begin{frame}
\frametitle{Geometric Distribution}
\Large
\begin{itemize}

\item Geometric distributions model (some) discrete random variables.
\item  Typically, a Geometric random variable is the number of trials required to obtain the first \textbf{success}.
\item For example, the number of tosses of a coin untill the first 'tail' is obtained, or a process where components from a production line are tested, in turn, until the first defective item is found.

\end{itemize}
\end{frame}
\begin{frame}
\frametitle{Geometric Distribution}
\Large
\begin{itemize}
\item A Geometric random variable is the number of trials until the first \textit{\textbf{success}}, whereas a Binomial random variable is the number of successes in $n$ trials.
\end{itemize}
\end{frame}
%-------------------------------------------------------------%
\begin{frame}
\frametitle{Geometric Distribution}
\Large
A discrete random variable X is said to follow a Geometric distribution with parameter p, written \[X \sim Geo(p),\] if it has probability distribution
\[P(X=x) = p^{x-1}(1-p)^x\]
where
\begin{itemize}
\item $x = 0, 1, 2, 3, \ldots$
\item p = success probability; $0 < p < 1$
\end{itemize}

\end{frame}
%-------------------------------------------------------------%
\begin{frame}
\frametitle{Geometric Distribution}
\Large
The trials must meet the following requirements:

\begin{itemize}
\item[(i)] the total number of trials is potentially infinite;
\item[(ii)] there are just two outcomes of each trial; success and failure;
\item[(iii)] the outcomes of all the trials are statistically independent;
\item[(iiv)] all the trials have the same probability of success.
\end{itemize}

\end{frame}
%-------------------------------------------------------------%
\begin{frame}
\frametitle{Geometric Distribution}
\Large
\begin{itemize}
\item The Geometric distribution has expected value and variance  \[E(X)= {1\over(1-p)}\] \[V(X)=\frac{p}{{(1-p)^2}}\].

\end{itemize}

\end{frame}
%-------------------------------------------------------------%
\begin{frame}
\frametitle{Geometric Distribution}
\Large
\begin{itemize}
\item The Geometric distribution is related to the Binomial distribution in that both are based on independent trials in which the probability of success is constant and equal to $p$. 

\item However, a Geometric random variable is the number of trials until the first \textit{\textbf{success}}, whereas a Binomial random variable is the number of successes in $n$ trials.
\end{itemize}
\end{frame}
%-------------------------------------------------------------%
\end{document}