\documentclass[a4]{beamer}
\usepackage{amssymb}
\usepackage{graphicx}
\usepackage{subfigure}
\usepackage{newlfont}
\usepackage{amsmath,amsthm,amsfonts}
%\usepackage{beamerthemesplit}
\usepackage{pgf,pgfarrows,pgfnodes,pgfautomata,pgfheaps,pgfshade}
\usepackage{mathptmx}  % Font Family
\usepackage{helvet}   % Font Family
\usepackage{color}

\mode<presentation> {
 \usetheme{Frankfurt} % was 
 \useinnertheme{rounded}
 \useoutertheme{infolines}
 \usefonttheme{serif}
 %\usecolortheme{wolverine}
% \usecolortheme{rose}
\usefonttheme{structurebold}
}

\setbeamercovered{dynamic}

\title[MA4413]{Statistics for Computing \\ {\normalsize MA4413 Lecture 4A}}
\author[Kevin O'Brien]{Kevin O'Brien \\ {\scriptsize Kevin.obrien@ul.ie}}
\date{Autumn Semester 2012}
\institute[Maths \& Stats]{Dept. of Mathematics \& Statistics, \\ University \textit{of} Limerick}

\renewcommand{\arraystretch}{1.5}

\begin{document}

\begin{frame}
\titlepage
\end{frame}

\frame{
\frametitle{The Binomial Probability Distribution}
A Quick Review of the Binomial Distribution
\begin{itemize}
\item The number of independent trials is denoted $n$.
\item The outcome of interest is known as a ``Success".
\item The other outcome is known as a ``failure".  
\item Often the applications of these names is counter-intuitive, i.e. defective components being the ``success".
\item The probability of a `success' is $p$ 
\item The expected number of `successes' from $n$ trials is $E(X) = np$
\item The \texttt{binom} family of commands in \texttt{R} are what we use to compute necessary values.
\end{itemize}
}

\end{document}



%---------------------------------------------------------------------------------------------------------------%
%----R Code ----
%---------------------------------------------------------------------------------------------------------------%
n=60000
Y=numeric(n)
for ( i in 1:n){

X=floor(runif(100,1,7))
Y[i]=sum(X)
}

Y
hist(Y,breaks=seq(300,400,by=10),main=c("Totals of 100 Die Throws"),cex.lab=1.4,font.lab=2,xlab=c("Total Score"))

hist(Y,breaks=seq(300,400,by=20),main=c("Totals of 100 Die Throws"),cex.lab=1.4,font.lab=2,xlab=c("Total Score"))



Z=seq(1:n)
Y/Z

plot(Y/Z,type="l",col="red",main=c("Die Rolls: Running Average"),font.lab=2,ylab="Average Value", xlab=
" Number of Throws")
abline(h=3.5,col="green")


#####################################################

plot(Z,Z.y,pch=16,col="red",ylim=c(2.5,5.5),main=c("Variance"),font.lab=2,ylab=" ", xlab="X: Green  Y: Blue  Z: Red" )

points(Y,Y.y,pch=16,col="blue" )
points(X,X.y,pch=16,col="green" )
points(c(1000,1000,1000),c(3,4,5),pch=18,cex=1.2)
lines(c(1000,1000),c(2.75,5.25),lty=3)



n=100000
Y=numeric(n)
for ( i in 1:n){

X=floor(runif(100,1,7))
Y[i]=sum(X)
}

Y
hist(Y,breaks=seq(270,430,by=2),main=c("Totals of 100 Die Throws (n= 100,000)"),cex.lab=1.4,font.lab=2,xlab=c("Total Score")) 
