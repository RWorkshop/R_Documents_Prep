# Hypothesis Testing and Inference Procedures in R

## `prop.test()`: Test for Proportions

Used to test hypotheses about proportions. Specify:
- Number of successes
- Total number of trials
- Expected probability under the null hypothesis

---

## Two-Tailed Test of Population Mean (Known Variance)

### Hypotheses
- **Null (H₀):** μ = μ₀
- **Alternative (H₁):** μ ≠ μ₀

### Test Statistic
\[
z = \frac{\bar{x} - \mu_0}{\sigma / \sqrt{n}}
\]

Reject H₀ if:
\[
z \leq -z_{\alpha/2} \quad \text{or} \quad z \geq z_{\alpha/2}
\]

### Example

- Previous mean weight: 15.4 kg
- Sample mean: 14.6 kg
- Population standard deviation: 2.5 kg
- Sample size: 35
- Significance level: α = 0.05

```r
xbar = 14.6
mu0 = 15.4
sigma = 2.5
n = 35
z = (xbar - mu0) / (sigma / sqrt(n))
z

alpha = 0.05
z.half.alpha = qnorm(1 - alpha / 2)
c(-z.half.alpha, z.half.alpha)
```

**Conclusion:** If z is between -1.9600 and 1.9600, fail to reject H₀.

---

## Paired t-Test: Matched Samples

Use for repeated measurements on the same subjects. Assume normality.

### Example: Barley Yields (Immer Dataset)

```r
library(MASS)
head(immer)

t.test(immer$Y1, immer$Y2, paired = TRUE)
```

**Output Summary:**
- t = 3.324
- df = 29
- p-value = 0.0024
- 95% CI: (6.122, 25.705)
- Mean difference ≈ 15.913

---

## `var.test()`: Test for Equality of Variances

Tests if two samples have equal variance. Essential for deciding whether to use the Welch or pooled t-test.

---

## t-Test Summary

Used to compare means between two groups. Variants include:

### 1. **Two-Sample t-Test (Welch’s Test)**  
Does *not* assume equal variance.

```r
t.test(var1, var2)
```

### 2. **Two-Sample t-Test (Pooled Variance)**  
Assumes equal variance.

```r
t.test(var1, var2, var.equal = TRUE)
```

### 3. **Paired t-Test**  
Used for matched pairs.

```r
t.test(var1, var2, paired = TRUE)
```

Each test returns:
- t-value
- degrees of freedom
- p-value
- confidence interval
- mean of each group or difference

---

## P-Values

P-values represent the probability of observing your test statistic or one more extreme **under the assumption that the null hypothesis is true**. They are **not** the probability that the null hypothesis is true—a commonly misunderstood distinction.

---

