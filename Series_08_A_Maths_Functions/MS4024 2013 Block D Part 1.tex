
\section{The R Programming Language}

The R Programming Language is a statistical , data analysis , etc

R is a free software environment for statistical computing and graphics.


\section{Vector types}
``R} operates on named data structures. The simplest such structure is the
vector, which is a single entity consisting of an ordered collection of
Numbers or characters.

\begin{itemize}
*  Numeric vectors
*  Character vectors
*  Logical vectors
*  (also complex number vectors and colour vectors)
\end{itemize}

To create a vector, use the assignment operator and the concatenate function.
For numeric vectors, the values are simply numbers.

<code>
># week8.r
>NumVec<-c(10.4,5.6,3.1,6.4)
</code>

Alternatively we can use the ``assign()`` command

For character vectors, the values are simply characters, specified with
quotation marks.A logical vectors is a vector whose elements are TRUE, FALSE or NA

<code>
>CharVec<-c(``blue", ``green", ``yellow")
>LogVec<-c(TRUE, FALSE)
</code>

\section{Graphical data entry interface}

``Data.entry()`` is a useful  command for inputting or editing data sets. Any
changes are saved automatically (i.e. don’t need to use the assignment
operator). We can also used the ``edit()`` command, which calls the ``R Editor}.

<code>
>data.entry(NumVec)
>NumVec <- edit(NumVec)
</code>

Another method of creating vectors is to use the following
<code>
numeric (length = n)
character (length = n)
logical (length = n)
</code>
These commands create empty vectors, of the appropriate kind, of length $n$. You can then use the graphical data entry interface to populate your data sets.

\subsubsection{Accessing specified elements of a vector}

The $n$th element of vector ``Vec" can be accessed by specifying its index when
calling ``Vec".
<code>>Vec[n]
</code>
A sequence of  elements of vector ``Vec" can be accessed by specifying its index
when calling ``Vec".
<code>>Vec[l:u]
</code>
Omitting and deleting the $n$th element of vector ``Vec"
<code>
>Vec[-n]
>Vec <- Vec[-n]
</code>

%\subsection{Reading data}


\subsection{inputting data}
Concatenation

\subsection{using help}

<pre>
<code>
?mean
</code>
</pre>

%\subsection{Adding comments}


\newpage
\section{Managing Precision}
\begin{itemize}
*  ``floor()`` - 
*  ``ceiling()`` - 
*  ``round()`` - 
*  ``as.integer()`` -
\end{itemize}

<pre>
<code>
> pi
[1] 3.141593
> floor(pi)
[1] 3
> ceiling(pi)
[1] 4
> round(pi,3)
[1] 3.142
> as.integer(pi)
[1] 3
</code>
</pre>

\subsection{exponentials, powers and logarithms}

<pre>
<code>
>x^y
>exp(x)
>log(x)
>log(y)
#determining the square root of x
>sqrt(x)
</code>
</pre>

%============================================================================================== %

\subsection{Packages}
The capabilities of ``R} are extended through user-submitted packages, which allow specialized statistical techniques, graphical devices, as well as and
import/export capabilities to many external data formats.




\subsection{vectors}

<pre>
<code>
R handles vector objects quite easily and intuitively.

> x<-c(1,3,2,10,5)    #create a vector x with 5 components
> x
[1]  1  3  2 10  5
> y<-1:5              #create a vector of consecutive integers
> y
[1] 1 2 3 4 5
> y+2                 #scalar addition
[1] 3 4 5 6 7
> 2*y                 #scalar multiplication
[1]  2  4  6  8 10
> y^2                 #raise each component to the second power
[1]  1  4  9 16 25
> 2^y                 #raise 2 to the first through fifth power
[1]  2  4  8 16 32
> y                   #y itself has not been unchanged
[1] 1 2 3 4 5
> y<-y*2
> y                   #it is now changed
[1]  2  4  6  8 10
</code>
</pre>


\subsubsection{Misc}
``seq()`` and ``rep()`` are useful commands for constructing vectors with a certain pattern.

%\end{document}

\subsection{Matrices}
A matrix refers to a numeric array of rows and columns.

One of the easiest ways to create a matrix is to combine vectors of equal
length using cbind(), meaning "column bind". Alternatively one can use rbind(), meaning ``row bind".



<pre>
<code>
> X=c(1,4,5,7,8,9,5,8,9)
> mean(X);median(X)       #mean and median of vector
[1] 6.222222
[2] 7
> sd(X)                   #standard deviation of Vector
[1] 2.682246
> length(X)               #sample size of vector
[1] 9
> sum(X)
[1] 56
> X^2
[1]  1 16 25 49 64 81 25 64 81
> rev(X)
[1] 9 8 5 9 8 7 5 4 1
> sort(X)                 #items in ascending order
[1] 1 4 5 5 7 8 8 9 9
> X[1:5]
[1] 1 4 5 7 8
</code>
</pre>



\section{Summary Statistics}
The ``R} command ``summary()`` returns a summary statistics for a simple dataset.
The ``R} command ``fivenum()`` returns a summary statistics for a simple dataset, but without the mean.
Also, the quartiles are computed a different way.


<pre>
<code>
> summary(mtcars$mpg)
Min. 1st Qu.  Median    Mean 3rd Qu.    Max.
10.40   15.43   19.20   20.09   22.80   33.90 
>
> fivenum(mtcars$mpg)
[1] 10.40 15.35 19.20 22.80 33.90
</code>
</pre>





\section{Bivariate Data}
 <pre>
<code>
> Y=mtcars$mpg
> X=mtcars$wt
>
> cor(X,Y)          #Correlation
[1] -0.8676594
>
> cov(X,Y)          #Covariance
[1] -5.116685
</code>
</pre>



