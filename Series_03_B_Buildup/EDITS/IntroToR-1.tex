 \documentclass{beamer}
 
 \usepackage{amsmath}
 \usepackage{graphics}
 \usepackage{graphicx}
 \usepackage{framed}
 \usepackage{amssymb}
 
 \begin{document}
 %==============================================================================================%
 
 % % SLIDE 1 - COVER SLIDE
 \begin{figure}
 \centering
 \includegraphics[width=0.9\linewidth]{Rlogo}
 \end{figure}
 \[ \mbox{Introduction to R} \]
 
 

 
 %============================================================================= %
 
 \frametitle{Introduction to R}
 \begin{itemize}
 \item[1.1] Installing R      
 \item[1.2] Command Line Interface     
 \item[1.3] The Assignment operator     
 \item[1.4] Commenting      
 \item[1.5] Defining Variables     
 \item[1.6] Help Functions      
 \item[1.7] The \texttt{help.start()} command     
 \item[1.8] Basic Maths Operations     
 \item[1.9] Basic R Editor      
 \end{itemize}
 
 
 %============================================================================= %
 
 \frametitle{Introduction to R}
 \begin{itemize}
 \item[1.10] Built-In Data Sets      
 \item[1.11] The \texttt{summary()} command     
 \item[1.12] Working directories      
 \item[1.13] Coming Unstuck    
 \item[1.14] Quitting the R environment   
 \item[1.15] Data Objects  
 \item[1.16] Listing all items in a workspace     
 \item[1.17] Removing items   
 \item[1.18] Saving and Loading R Data Objects    
 \end{itemize}
 
 
 %============================================================================= %
 
 \frametitle{Introduction to R (Continued) }
 \begin{itemize}
 \item[2.1] Three particularly useful commands    
 \item[2.2] Changing GUI options     
 \item[2.3] Colours      
 \item[2.4] Use of the Semi-Colon Operator     
 \item[2.5] The \texttt{apropos()} Function     
 \item[2.6] History       
 \item[2.7] The \texttt{sessionInfo()} Function     
 \item[2.8] Time and date functions     
 \item[2.9] Logical States      
 \item[2.10] Missing Data      
 \item[2.11] Files in the Working Directory     
 \end{itemize}
 
 %============================================================================= %
 
 \frametitle{Section 3 - Inspecting a Data Set }
 \begin{itemize}
 \item[3.1] Dimensions of a data set 
 \item[3.2] The \texttt{summary()} command 
 \item[3.3] Structure of a Data Object 
 \end{itemize}
 
 
 % %============================================================================= %
 %
 %\frametitle{Section 4 : Packages}
 %
 % \begin{semiverbatim}
 % 4.1 Packages 
 % 4.2 Using and Installing packages 
 % 4.2.1 Version of R 
 % \end{semiverbatim}
 %
 % 
 %
 %\end{document}
 %============================================================================= %
 
 \frametitle{Part 5 - Data Creation, Data Entry, Data Import and Export}
 \begin{framed}
 \begin{semiverbatim}
 5.1 The c() command 
 5.1.1 Vector of Numeric Values
 5.1.2 Vector of Character Values
 5.1.3 Vector of Logical Values 
 5.2 The scan() command 
 5.2.1 Characters with the scan() command
 5.3 Using the data editor
 5.4 Spreadsheet Interface 
 \end{semiverbatim}
 \end{framed}
 
 
 
 
 %==============================================================================================%
 
 \frametitle{Introduction to R}
 Source: R project website (http://www.r-project.org)
 \begin{itemize}
 \item R is a language and environment for statistical computing and graphics. It is a GNU project
 which is similar to the S language and environment which was developed at Bell Laboratories
 (formerly AT\&T, now Lucent Technologies) by John Chambers and colleagues. 
 \item R can be considered
 as a different implementation of S. There are some important differences, but much
 code written for S runs unaltered under R.
 \end{itemize}
 
 
 
 %==============================================================================================%
 
 \frametitle{What is R?}
 \begin{itemize}
 \item R provides a wide variety of statistical (linear and nonlinear modelling, classical statistical tests,
 time-series analysis, classification, clustering, ...) and graphical techniques, and is highly extensible.
 The S language is often the vehicle of choice for research in statistical methodology,
 and \texttt{R} provides an Open Source route to participation in that activity.
 \item One of \texttt{R}’s strengths is the ease with which well-designed publication-quality plots can be
 produced, including mathematical symbols and formulae where needed. 
 \item Great care has been
 taken over the defaults for the minor design choices in graphics, but the user retains full control.
 \item \texttt{R} is available as Free Software under the terms of the Free Software Foundation’s GNU General
 Public License in source code form. It compiles and runs on a wide variety of UNIX platforms
 and similar systems (including FreeBSD and Linux), Windows and MacOS.
 \end{itemize}
 
 %==============================================================================================%
 
 
 \texttt{R} is a programming environment that
 \begin{itemize}
 \item uses a well-developed but simple programming language
 \item allows for rapid development of new tools according to user demand
 \item these tools are distributed as packages, which any user can download to customize the R
 environment.
 \end{itemize}
 
 
 
 %==============================================================================================%
 
 % % SLIDE 1 - COVER SLIDE
 \begin{figure}
 \centering
 \includegraphics[width=0.7\linewidth]{images/Rhelpcommands}
 %\caption{}
 %\label{fig:Rhelpcommands}
 \end{figure}
    
 
 %==============================================================================================%
 
 % % SLIDE 1 - COVER SLIDE
 \begin{figure}
 \centering
 \includegraphics[width=0.7\linewidth]{images/rabbitsummary}   
 \end{figure}
  
 
 
 %==============================================================================================%
 
 % % SLIDE 1 - COVER SLIDE
 \begin{figure}  
 \includegraphics[width=0.7\linewidth]{images/irisinspect}     
 \end{figure}
    
 %==============================================================================================%
 
 % % SLIDE 1 - COVER SLIDE
\begin{figure}
\centering
\includegraphics[width=0.7\linewidth]{images/irissummary}
\caption{}
\label{fig:irissummary}
\end{figure}
    
 
 
 %==============================================================================================%
 
 \frametitle{Comprehensive R Archive Network}
 \begin{itemize}
 \item Base R and most R packages are available for download from the Comprehensive R Archive Network
 (CRAN) cran.r-project.org. 
 \item Base R comes with a number of basic data management,
 analysis, and graphical tools 
 \item R’s power and flexibility, however, lie in its array of packages
 (currently more than 6,000!)
 \end{itemize}
 
 
 %==============================================================================================%
 
 \frametitle{1.1 Installing R}
 \begin{itemize}
 \item \texttt{R} is very easily installed by downloading it from the CRAN website. Installation usually takes
 about 2 minutes. 
 \item When Installation of R is complete, the distinctive R Icon will appear on your
 desktop. To start \texttt{R}, simply click this Icon. 
 \item It is common to re-install \texttt{R} once a year or so. The
 current version of \texttt{R}, version 3.1.2 was released quite recently.
 \end{itemize}
 
 
 
 
 %==============================================================================================%
 
 % % SLIDE 1 - COVER SLIDE
 \begin{figure}
 \centering
 \includegraphics[width=0.7\linewidth]{images/Rversion}        
 \end{figure}
   
 %==============================================================================================%
 
 
 \frametitle{1.2 Command Line Interface}
 \begin{itemize}
 \item When you start \texttt{R}, the command line interface window will appear on screen. This is one
 of many windows in the \texttt{R} environment, others including graphical output windows, or script
 editors. 
 \item \texttt{R} code can be entered into the command line directly. 
 \item Alternatively code can be saved
 to a script, which can be run inside a session using the \texttt{source()} function.
 \end{itemize}
 
 %==============================================================================================%
 
 \frametitle{1.3 The Assignment operator}
 \begin{itemize}
 \item The assignment operator is used to assign names to particular values. 
 \item Historically the assignment
 operator was ) a ``\texttt{<-}”. 
 \item The assignment operator can also be the equals sign "=". (This is valid as of \texttt{R}
 version 1.4.0.)
 
 \item Both will be used, although, you should learn one and stick with it. Many long term \texttt{R}
 users prefer the arrow approach. 
 \end{itemize}
 
 
 %==============================================================================================%
 
 \frametitle{1.3 The Assignment operator}
 
 You can also use -> as an assignment operator, reversing the
 usual assignment positions. (This is actually really useful). Commands are separated either by
 a semi colon or by a newline.
 \begin{framed}
 \begin{semiverbatim}
 > a <- 6
 > b = 5
 > a + b ->c
 > c
 [1] 11
 >e=7;f<-4
 \end{semiverbatim}
 \end{framed}
 
 \end{document} 
 %==============================================================================================%
 
 \frametitle{1.3 The Assignment operator}
 \textbf{The Up and Down Keys}
 \begin{itemize}
 \itemBefore we continue, try using the up and down keys, and see what happens. 
 \item Previously
 typed commands are re-presented, and can be re-executed.
 \end{itemize}
 
 
 %==============================================================================================%
 
 \frametitle{1.3 The Assignment operator }
 \textbf{objects}
 \begin{itemize}
 \item R stores both data and output from data analysis (as well as everything else) in \textbf{objects}.
 \item The variables we have created so far are objects. 
 \item A list of all objects in the current session can
 be obtained with ls().
 \end{itemize}
 
 
 %==============================================================================================%
 
 
 \textbf{1.3.1 Reserved Words - Bad names for Objects}
 \begin{itemize}
 \item Some names are used by the system, e.g.T, F,q,c etc . 
 \item Avoid using these.
 \item Also avoide using command names like \textbf{mean} and \textbf{sum}
 \end{itemize}
 
 %==============================================================================================%
 
 % % SLIDE 1 - COVER SLIDE
 \begin{figure}
 \centering
 \includegraphics[width=0.7\linewidth]{images/typeconversion} 
 \end{figure}
    
 
 
 
 %==============================================================================================%
 
 % % SLIDE 1 - COVER SLIDE
 \begin{figure}
 \centering
 \includegraphics[width=0.7\linewidth]{images/numerictypes}    
 \end{figure}
    
 %==============================================================================================%
 
 \frametitle{1.4 Commenting}
 For the sake of readability, it is good practice to comment code. The \# character at the
 beginning of a line signifies a comment, which is not executed. Lines of hashtags can be used
 to identify the beginning and end of code segments
 \begin{framed}
 \begin{semiverbatim}
 # This is a comment
 #####################
 # Start of Segment 1
 #####################
 \end{semiverbatim}
 \end{framed}
 
 %==============================================================================================%
 
 \frametitle{1.5 Defining and Naming Variables}
 \begin{itemize}
 \item A convention is to use define a variable name with a capital letter (R is case sensitive). 
 \item This
 reduces the chance of overwriting in-build R functions, which are usually written in lowercase
 letters. 
 \item Commonly used variable names such as x,y and z (in lower case letters) are not “reserved”.
 \end{itemize}
 
 \begin{semiverbatim}
 camelCase
 
 variableName
 
 AlsoCamelCase
 \end{semiverbatim}
 
 %==============================================================================================%
 
 % % SLIDE 1 - COVER SLIDE
 \begin{figure}
 \centering
 \includegraphics[width=0.7\linewidth]{images/Rmultiplewindows}
 \end{figure}
 
    
 %==============================================================================================%
 
 % % SLIDE 1 - COVER SLIDE
 \begin{figure}
 \centering
 \includegraphics[width=0.7\linewidth]{images/Rscript}         
 \end{figure}
    
 %==============================================================================================%
 
 
 \frametitle{1.6 Help Functions}
 \begin{itemize}
 \item Help files for R functions are accessed by preceding the name of the function with ? (e.g. ?sort
 ). 
 
 \item Alternatively you can use the command \texttt{help()} (e.g. help(sqrt) )
 \end{itemize}
 
 
 %==============================================================================================%
 
 
 \frametitle{1.6 Help Functions}
 \begin{itemize}
 \item A HTML document appears on screen with information on the function typed in. 
 \item As well
 as the list of arguments that the particular function accepts, and how to specify them, there is
 example code at the bottom of the file. 
 \item These code segments are often invaluable in learning
 how to master the function.
 \end{itemize}
 
 
 
 
 
 %==============================================================================================%
 
 % % SLIDE 1 - COVER SLIDE
 \begin{figure}
 \centering
 \includegraphics[width=0.7\linewidth]{images/Rapropos1}       
 \end{figure}
    
 %==============================================================================================%
 
 % % SLIDE 1 - COVER SLIDE
 \begin{figure}
 \centering
 \includegraphics[width=0.7\linewidth]{images/Rapropos2}       
 \end{figure}
    
 %==============================================================================================%
 
 
 
 \frametitle{1.7 The help.start() command}
 As mentioned by the text on the main console, the \texttt{help.start()} command can be usd to
 access detailed help documentation that comes as part of the installation.
 
 %==============================================================================================%
 
 
 \frametitle{1.8 Basic Maths Operations}
 The most commonly used mathematical operators are all supported by R. \\Here are a few
 examples:
 
 \begin{semiverbatim}
 5 + 3 * 5 # Note the order of operations.
 log (10) # Natural logarithm with base e=2.718282
 log(8,2) # Log to the base 2
 4^2 # 4 raised to the second power
 7/2 # Division
 factorial(4) #Factorial of Four
 sqrt (25) # Square root
 abs (3-7) # Absolute value of 3-7
 pi # The mysterious number \\\
 exp(2) # exponential function
 \end{semiverbatim}
 
 %==============================================================================================%
 
 
 \texttt{R} can be used for many mathematical operations, including
 
 \begin{itemize}
 \item Set Theory
 \item Trigonometry
 \item Complex Numbers
 \item Binomial Coefficients
 \end{itemize}
 Set Theory is always useful to know. We will not go into any of the others in great detail today.
 
 %==============================================================================================%
 
 
 1.9 Basic R Editor
 R has an inbuilt script editor. We will use it for this class, but there are plenty of top quality
 Integrated Development Environments out there. (Read up about RStudio for example).
 For a while, we will use the in-built script editor. Although it is advisable to start using Rstudio or something similar.
 
 
 %==============================================================================================%
 
 
 
 To start a new script, or open an existing script simply go to File and click the appropriate
 options. A new dialogue box will appear. You can write and edit code using this editor.
 To pass the code for compiling , press the run line or selection option (The third icon
 on the menu).
 
 
 %==============================================================================================%
 
 
 Another way to read code is to use the edit() function , which operates directly from the
 command line. To see how the code defining an object X was written (or more importantly,
 could have been written) simply type \texttt{edit(X)}. This command has some useful applications
 that we will see later on.
 
 
 %==============================================================================================%
 
 
 \textbf{Script Files}
 \begin{itemize}
 \item Scripts are saved as \texttt{.R} files. 
 \item These scripts can be called directly using the \texttt{source()} command.
 \end{itemize}
 
 %==============================================================================================%
 
 \frametitle{1.10 Built-In Data Sets}
 \textbf{Inbuilt Data Sets}\\
 Several data sets , intended as learning tools, are automatically installed when R is installed.
 Many more are installed within packages to complement learning to use those packages. One
 of these is the famous Iris data set, which is used in many data mining exercises.
 
 \begin{itemize}
 \item iris
 \item mtcars
 \item Nile
 \end{itemize}
 
 
 
 %==============================================================================================%
 
 % % SLIDE 1 - COVER SLIDE
 \begin{figure}
 \centering
 \includegraphics[width=0.7\linewidth]{images/Rdatasets}        
 \end{figure}
    
 %==============================================================================================%
 
 % % SLIDE 1 - COVER SLIDE
 \begin{figure}
 \centering
 \includegraphics[width=0.7\linewidth]{images/RdatasetsMore}   
 \end{figure}
    
 %==============================================================================================%
 
 
 \begin{itemize}
 \item To see what data sets are available, simply type \texttt{data()}.
 \item  To load a data set, simply type in the
 name of the data set. Some data sets are very large.
 \item  To just see the first few (or last) rows, we
 use the \texttt{head()} function or alternatively the \texttt{tail()} function. 
 \item The default number of rows of
 these commands is 6. Other numbers can be specified.
 \end{itemize}
 
 
 %==============================================================================================%
 
 % % SLIDE 1 - COVER SLIDE
 \begin{figure}
 \centering 
 \includegraphics[width=0.7\linewidth]{images/irishead}      
 \end{figure}
    
 
 %==============================================================================================%
 
 % % SLIDE 1 - COVER SLIDE
 \begin{figure}
 \centering
 \includegraphics[width=0.7\linewidth]{images/iristail}     
 \end{figure}
    
 %==============================================================================================%
 
 \begin{itemize}
 \item In many situations, it is useful to call a particular data set using the \texttt{attach()} command. This
 will save having to specify the data sets over repeated operations. 
 \item The file can then be detached
 using the \texttt{detach()} command.
 \end{itemize}
 
 
 
 
 %==============================================================================================%
 
 \frametitle{1.11 The summary() command}
 The R command summary() is a generic function used to produce result “summaries” of the
 results of various objects, from simple vectors to the output of complex model fitting functions.
 The function invokes particular methods which depend on the class of the first argument.
 > summary(Nile)
 Min. 1st Qu. Median Mean 3rd Qu. Max.
 456.0 798.5 893.5 919.4 1032.0 1370.0
 
 %==============================================================================================%
 
 > summary(Indometh)
 Subject time conc
 1:11 Min. :0.250 Min. :0.0500
 4:11 1st Qu.:0.750 1st Qu.:0.1100
 2:11 Median :2.000 Median :0.3400
 5:11 Mean :2.886 Mean :0.5918
 6:11 3rd Qu.:5.000 3rd Qu.:0.8325
 3:11 Max. :8.000 Max. :2.7200
 
 %==============================================================================================%
 
 1.12 Working directories
 You can change your working directly by using the appropriate options on the File menu. To
 determine the current working directory, you can use the getwd() command. To change the
 working directory , we would use the setwd() command. This is quite important as objects
 will be imported and exported to and from the specified directory.
 
 %==============================================================================================%
 
 > getwd()
 [1] "C:/Users/Kevin"
 >
 > setwd("C:/Users/Kevin/Documents")
 >
 > getwd()
 [1] "C:/Users/Kevin/Documents"
 
 %==============================================================================================%
 
 \frametitle{1.13 Coming Unstuck}
 \Large
 \begin{itemize}
 \itemIf you are having trouble with a piece of code that is currently compiling , all you have to do is press ESC, just like many other computing environments.
 \end{itemize}
 
 %==============================================================================================%
 
 \frametitle{1.14 Quitting the R environment}
 As the front page text indicates, all you have to do to quite the workspace is to type in \texttt{q()}.
 You will then be prompted to save your work.
 
 %==============================================================================================%
 
 \frametitle{1.15 Data Objects}
 As mentioned previously, R saves data as \textbf{objects}. Examples of data objects are
 \begin{itemize}
 \item Vectors
 \item Lists
 \item Dataframes
 \item Matrices
 \end{itemize}
 The simple objects we have created previously are simply single element vectors.
 
 %==============================================================================================%
 
 \frametitle{1.16 Listing all items in a workspace}
 To list all items in an R environment, we use the \texttt{ls()} function. This provides a list of all data
 objects accessible. Another useful command is \texttt{objects()}.
 \begin{framed}
 \begin{semiverbatim}
 > ls()
 [1] "a" "A" "authors" "b" "books"
 [6] "C" "D" "ex1" "Gerb" "Lst"
 [11] "m" "m1" "op" "presidents" "r"
 [16] "showSmooth" "sm" "sm.3RS" "sm2" "sm3"
 [21] "Trig" "Vec1" "x" "X" "x.at"
 [26] "x1" "x2" "x3R" "y" "Y"
 [31] "y.at"
 \end{semiverbatim}
 \end{framed}
 
 %==============================================================================================%
 
 \frametitle{1.17 Removing items}
 \begin{itemize}
 \item Sometimes it is desirable to save a subset of your workspace instead of the entire workspace.
 \item One option is to use the \texttt{rm()} function to remove unwanted objects right before exiting your R
 session; another possibility is to use the \texttt{save()} function.
 \end{itemize}
 
 %==============================================================================================%
 
 \frametitle{1.18 Saving and Loading R Data Objects}
 In situations where a good deal of processing must be used on a raw dataset in order to prepare
 it for analysis, it may be prudent to save the R objects you create in their internal binary form.
 One attractive feature of this scheme is that the objects created can be read by R programs
 running on different computer architectures than the one on which they were created, making it
 very easy to move your data between different computers. Each time an R session is completed,
 you are prompted to save the workspace image, which is a binary file called .RData in the
 working directory.
 
 %==============================================================================================%
 
 Whenever R encounters such a file in the working directory at the beginning of a session,
 it automatically loads it making all your saved objects available again. So one method for
 
 saving your work is to always save your workspace image at the end of an R session. If you
 would like to save your workspace image at some other time during your R session, you can use
 the save.image() function, which, when called with no arguments, will also save the current
 workspace to a file called .RData in the working directory.
 
 
 
 %==============================================================================================%
 
 %2 Introduction to R (Continued)
 \frametitle{2.1 Three particularly useful commands}
 
 \begin{itemize}
 \item \texttt{help()}
 \item \texttt{summary()}
 \item \texttt{help.start()}
 \end{itemize}
 
 
 %==============================================================================================%
 
 \frametitle{2.2 Changing GUI options}
 \begin{itemize}
 \item We can change the GUI options using the GUI preferences option on the Edit menu.
 \item  (Important
 when teaching R) 
 \item A demonstration will be done in class.
 \end{itemize}
 
 
 %==============================================================================================%
 
 \frametitle{2.3 Colours}
 \begin{itemize}
 \item R supported a massive number of colours.
 \item Type in colours() (or colors()) to see what colours
 are supported.
 \end{itemize}
 
 %==============================================================================================%
 
 \frametitle{2.4 Use of the Semi-Colon Operator}
 \begin{itemize}
 \item The semi-colon operator at the end of each line of code is not necessary in general, but using it
 overcomes errors due to copying and pasting from document soft copies. 
 \item It is also useful for compacting multiple short statements onto a single line.
 \item In other programming
 languages, such as Octave, using the semicolon in this way has a distinct purpose.
 \end{itemize}
 
 %==============================================================================================%
 
 \frametitle{2.5 The apropos() Function}
 This function is very useful for determining what functions are available for a particular topic,
 although the process requires a great deal of trial and error. Suppose we are looking for a
 command to compute the correlation coefficient. We would use a very short string (e.g. cor)
 that would plausibly be part of useful function names.
 
 
 %==============================================================================================%
 
 \frametitle{2.6 History}
 \begin{itemize}
 \item The command \texttt{history()} is used to obtain the last 25 commands used by \texttt{R}.
 \item 25 is the default number, you can specify another number.
 \end{itemize}
 \begin{framed}
 \begin{semiverbatim}
 history(100)
 \end{semiverbatim}
 \end{framed}
 
 
 %==============================================================================================%
 
 % % SLIDE 1 - COVER SLIDE
 \begin{figure}
 \centering
 \includegraphics[width=0.7\linewidth]{images/Rhistory}        
 \end{figure}
    
 %==============================================================================================%
 
 \frametitle{2.7 The \texttt{sessionInfo()} Function}
 To get a description of the version of R and its attached packages used in the current session,
 we can use the \texttt{sessionInfo()} function
 
 
 %==============================================================================================%
 
 \frametitle{2.8 Time and date functions}
 The commands Sys.time() and Sys.Date() returns the system’s idea of the current date
 with and without time. We can perform some simple algebraic calculations to compute time
 differences (i.e. to find out how long some code took to compile.
 
 
 %==============================================================================================%
 
 \frametitle{System Time}
 \begin{framed}
 \begin{semiverbatim}
 > X1=Sys.time()
 > #Wait a few seconds
 >
 > X2=Sys.time()
 > X2-X1 Time difference of 8.439614 secs
 >
 > Sys.Date() [1] "2012-09-01"
 \end{semiverbatim}
 \end{framed}
 
 %==============================================================================================%
 
 \frametitle{2.9 Logical States}
 \begin{itemize}
 \item Logical states TRUE and FALSE are simply specified as such, all in capital letters. 
 \item The
 shortcuts T and F are also recognized
 \end{itemize}
 
 %==============================================================================================%
 
 
 
 2.10 Missing Data
 In some cases the entire contents of a vector may not be known. For example, missing data
 from a particular data set. A place can be reserved for this by assigning it the special value
 NA.
 NA is a logical constant of length 1 which contains a missing value indicator. NA stands
 for Not Available.
 
 
 %==============================================================================================%
 
 
 \frametitle{2.11 Files in the Working Directory}
 It is possibel to call an R script from the working directory, using the \texttt{source()} function.
 \begin{framed}
 \begin{semiverbatim}
 source("myfunctions.r")
 source("mydata.r")
 \end{semiverbatim}
 \end{framed}
 We can also send code put directly to a file in the working directory, using the sink()
 command. The first time the command is used, the name of the created file is specified.
 Subsequent commands print output directly to this file, until the command is used again to
 cease the operation.
 
 %==============================================================================================%
 
 \begin{framed}
 \begin{semiverbatim}
 > sink("IrisSum.txt")
 > summary(iris)
 > sink()
 >
 \end{semiverbatim}
 \end{framed}
 
 
 %==============================================================================================%
 
 
 \Huge
 \[\mbox{ Section 3 : Inspecting a Data Set } \]
 
 %==============================================================================================%
 
 \frametitle{Section 3 Inspecting a Data Set}
 \large 
 \textbf{Summary of useful commands}
 \begin{itemize}
 \item \texttt{dim()} and \texttt{length()}
 \item \texttt{nrow()} and \texttt{ncol()}
 \item \texttt{names()}
 \item \texttt{summary()}
 \item \texttt{tail()}
 \item \texttt{head()}
 \item \texttt{describe()} (from the \textbf{psych} package)
 \end{itemize}
 
 %==============================================================================================%
 
 
 \frametitle{3.1 Dimensions of a data set}
 \begin{itemize}
 \item We have remarked that some data sets are very large. 
 \item This is perhaps a good place to consider
 summary information about data objects. 
 \item For a simple vector, a useful command to determine
 the length (remark: sample size) is the function \texttt{length()}.
 \end{itemize}
 \begin{framed}
 \begin{semiverbatim}
 > Y=4:18
 > length(Y)
 [1] 15
 \end{semiverbatim}
 \end{framed}
 For more complex data sets ( and data frames which we will see later) , we have several
 tools for assessing the size of data.
 
 %==============================================================================================%
 
 \begin{framed}
 \begin{semiverbatim}
 > dim(iris) # dimensions of data set
 [1] 150 5
 > nrow(iris) # number of rows
 [1] 150
 > ncol(iris) # number of columns
 [1] 5
 \end{semiverbatim}
 \end{framed}
 
 
 %==============================================================================================%
 
 \frametitle{Column (Variable) names and Row names}
 \begin{itemize}
 \item We can also determine the row names and column names using the functions \texttt{rownames()}
 and \texttt{colnames()}. 
 \item If there are no specific row or column names, the command will just return
 the indices.
 \end{itemize}
 \begin{semiverbatim}
 > colnames(iris)
 [1] "Sepal.Length" "Sepal.Width" "Petal.Length" "Petal.Width" "Species"
 \end{semiverbatim}
 
 %==============================================================================================%
 
 \frametitle{3.2 The \texttt{summary()} command}
 \begin{itemize}
 \item The command \texttt{summary()} is one of the most useful commands in \texttt{R}. 
 \item It is a generic function used
 to produce result summaries of the results of various functions. 
 \item The function invokes particular
 methods which depend on the class of the first argument. 
 \item In other words, \texttt{R} picks out the most
 suitable type of summary for that data.
 \end{itemize}
 
 %==============================================================================================%
 
 
 
 \begin{figure}
 \centering
 \includegraphics[width=0.99\linewidth]{images/irissummary}
 
 \end{figure}
 \texttt{summary()} is particularly useful for manipulating data from more complex data objects.
 
 
 %==============================================================================================%
 
 
 \frametitle{3.3 Structure of a Data Object}
 \large
 The structure, class and storage mode of an object can be determined using the following
 commands. Try out a few.
 \begin{itemize}
 \item  \texttt{str()}
 \item  \texttt{class()}
 \item  \texttt{mode()}
 \end{itemize}
 
 
 
 %==============================================================================================%
 
 \begin{framed}
 \begin{semiverbatim}
 > class(Nile)
 [1] "ts"
 > mode(Nile)
 [1] "numeric"
 >
 
 \end{semiverbatim}
 \end{framed}
 
 
 %==============================================================================================%
 
 \begin{semiverbatim}
 > a
 [1] 6
 > mode(a)
 [1] "numeric"
 > class(a)
 [1] "numeric"
 > str(a)
 num 6
 >
 > class(iris)
 [1] "data.frame"
 > mode(iris)
 [1] "list"
 \end{semiverbatim}
 
 
 %==============================================================================================%
 
 
 \Huge
 \[\mbox{ Section 4 : Packages } \]
 
 
 \frametitle{Packages}
 
 \begin{itemize}
 \item A Package in \texttt{R} is a file containing a collection of objects which have some common purpose.
 \item Packages enhance the capabilties and scope for research in a certain field. 
 \item For example, the
 package MASS contains objects associated with the Venables and Ripleys ``\textit{Modern Applied
 Statistics with S}”. 
 \end{itemize}
 
 
 %=================================================================== %
 
 
 
 \begin{figure}
 \centering
 \includegraphics[width=0.97\linewidth]{CRAN}
 \caption{Comprehensive R Archive Network}
 
 \end{figure}
 
 
 
 
 
 %=================================================================== %
 
 \frametitle{R Packages}
 
 \begin{itemize}
 \item ``10 R packages I wish I knew about earlier" - Drew Conway (Yhat.com, February 2013)
 \bigskip \item ``The HadleyVerse" - Hadley Wickham
 \begin{itemize}
 
 \item  ggplot2, dplyr, reshape, lubridate, stringr
 
 \item  With Romain Francois, Dianne Cook and Garret Grolemund.
 \end{itemize}
 \bigskip
 \item Dr Brendan Haplin (UL): lme4 ,TraMineR, Gelman's arm, MASS, foreign. 
 \bigskip
 \item Shiny - Web Applications with \texttt{R}
 \end{itemize}
 
 %=================================================================== %
 
 \frametitle{R Packages}
 
 
 
 
 Some examples of packages are Actuar, written for actuarial science, and
 QRMlib, which complements the Quantitative Risk Management The command library()
 lists all the available packages. 
 
 To load a particular package, for example MASS, we would
 write
 library(MASS)
 
 
 %=================================================================== %
 
 
 \frametitle{Packages}
 \begin{itemize}
 \item The CRAN package repository features 6107 available packages. 
 \item Packages contain
 various functions and data sets for numerous purposes, e.g.
 \textbf{\textit{ggplot2}} package, \textbf{\textit{AER}} package, \textbf{\textit{survival}} package, etc.
 \item Some packages are part of the basic installation. Others can be
 downloaded from CRAN.
 \item To access all of the functions and data sets in a particular package,
 it must be loaded into the workspace. 
 \item For example, to load the
 \textbf{\textit{ggplot2}} package:
 \end{itemize}
 \begin{framed}
 \begin{semiverbatim}
 install.packages(ggplot2)
 library(ggplot2)
 \end{semiverbatim}
 \end{framed}
 
 %==============================================================================================%
 
 \frametitle{4.2 Using and Installing packages}
 \begin{itemize}
 \item Many packages come with R. To use them in an R session, you need to load the package, as
 previously demonstrated.
 \item Some packages are not automatically installed when you install R but they need to be downloaded
 and installed individually. 
 \item We must first install them using the install.packages()
 function, which typically downloads the package from CRAN and installs it for use. (follow the
 instructions as indicated).
 \end{itemize}
 
 %==============================================================================================%
 
 \begin{framed}
 \begin{semiverbatim}
 install.packages("ggplot2")
 install.packages("qcc")
 install.packages("sqldf")
 install.packages("RMongo")
 install.packages("randomforest")
 \end{semiverbatim}
 \end{framed}
 
 
 %==============================================================================================%
 
 \frametitle{4.2.1 Version of R}
 Many packages will require you to have the most recent version of R and also other packages.
 It is a good idea to update regularly.
 
 %==============================================================================================%
 
 5 Data Creation, Data Entry, Data Import and Export
 
 %==============================================================================================%
 
 \frametitle{5.1 The \texttt{c()} command}
 To create a simple data set, known as a vector, we use the c() command to create the vector.
 \begin{framed}
 \begin{semiverbatim}
 > Y=c(1,2,4,8,16 ) #create a data vector with specified elements
 > Y
 [1] 1 2 4 8 16
 
 \end{semiverbatim}
 \end{framed}
 
 
 %==============================================================================================%
 
 \frametitle{Vectors}
 \begin{framed}
 \begin{semiverbatim}
 ### Vector of Numeric Values
 Numvec = c(10,13,15,19,25);
 ## Vector of Character Values
 Charvec = c("LouLou","Oscar","Rasher");
 
 ## Vector of Logical Values
 Charvec = c(TRUE,TRUE,FALSE,TRUE);
 \end{semiverbatim}
 \end{framed}
 
 %==============================================================================================%
 
 \frametitle{Vectors}
 Vectors can be bound together either by row or by column.
 \begin{framed}
 \begin{semiverbatim}
 > X=1:3; Y=4:6
 > cbind(X,Y)
 X Y
 [1,] 1 4
 [2,] 2 5
 [3,] 3 6
 >
 > rbind(X,Y)
 [,1] [,2] [,3]
 X 1 2 3
 Y 4 5 6
 \end{semiverbatim}
 \end{framed}
 
 %==============================================================================================%
 
 \frametitle{5.2 The scan() command}
 \begin{itemize}
 \itemThe \texttt{scan()} function is a useful method of inputting data quickly. 
 \item You can use to quickly copy
 and paste values into the \texttt{R} environment. It is best used in the manner as described in the
 following example. 
 \item Create a variable ”X” and use the \texttt{scan()} function to populate it with
 values. 
 \item Type in a value, and then press return. Once you have entered all the values, press
 return again to return to normal operation.
 \end{itemize}
 
 %==============================================================================================%
 
 \begin{semiverbatim}
 > X=scan()
 1: 4
 2: 5
 3: 5
 4: 6
 5:
 Read 4 items
 \end{semiverbatim}
 Remark: Try the \texttt{edit()} command on object X.
 
 %==============================================================================================%
 
 5.2.1 Characters with the scan() command
 The scan() command can also be used forinputting character data.
 > Y=scan(what=" ")
 1: LouLou
 2: Oscar
 3: Rasher
 4:
 Read 3 items
 > Y
 [1] "LouLou" "Oscar" "Rasher"
 
 %==============================================================================================%
 
 5.3 Using the data editor
 
 
 %==============================================================================================%
 
 5.4 Spreadsheet Interface
 R provides a spreadsheet interface for editing the values of existing data sets. We use the
 command \texttt{data.entry()}, and name of the data object as the argument. (Also try out the
 fix() command)
 \begin{framed}
 \begin{semiverbatim}
 > data.entry(X) # Edit the data set and exit interface
 > X
 \end{semiverbatim}
 \end{framed}
 
 
 
 
 
 
 
\end{document}
