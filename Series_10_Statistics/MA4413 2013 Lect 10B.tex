

Using <tt>R</tt> for Inference Procedures}
}
* [1] Review of the Paired t-test.
* [2] Paired t-test using <tt>R</tt>
%* [3] Two Sample Test for Proportions
* [3] Test for the equality of variances for two samples
* [4] Shapiro-Wilk Test for Normality
* [5] Graphical procedures for assessing normality
* [6] Grubb's Procedure for Determinin an Outlier




p-values using <tt>R</tt>}
}
*  In every inference procedure performed using <tt>R</tt>, a p-value is presented to the screen for the user to interpret.

*  If the p-value is larger than a specified threshold $\alpha/k$ then the appropriate conclusion is a
failure to reject the null hypothesis.

*  Conversely, if the p-value is less than threshold, the appropriate conclusion is to reject the null hypothesis.

*  In this module, we will use a significance level$\alpha=0.05$ and almost always the procedures will be two tailed ($k=2$). Therefore the threshold $\alpha/k$ will be $0.025$.





Using Confidence Limits}
}
*  Alternatively, we can use the confidence interval to make a decision on whether or not we should reject or fail to reject the null hypothesis.
*  If the null value is within the range of the confidence limits, we fail to reject the null hypothesis.
*  If the null value is outside the range of the confidence limits, we reject the null hypothesis.
*  Occasionally a conclusion based on this approach may differ from a conclusion based on the p-value. In such a case, remark upon this discrepancy.




----------------------------

