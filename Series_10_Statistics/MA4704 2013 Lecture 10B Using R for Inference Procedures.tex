
p-values using <tt>R</tt>}
}
*  In every inference procedure performed using <tt>R</tt>, a p-value is presented to the screen for the user to interpret.

*  \textbf{IMPORTANT} If the p-value is larger than a specified threshold $\alpha/k$ then the appropriate conclusion is a
failure to reject the null hypothesis.

*  \textbf{IMPORTANT} Conversely, if the p-value is less than threshold, the appropriate conclusion is to reject the null hypothesis.

*  In this module, we will use a significance level$\alpha=0.05$ and almost always the procedures will be two tailed ($k=2$). Therefore the threshold $\alpha/k$ will be $0.025$.
*  \textbf{Exception:} The Shapiro-Wilk Test for Normality is one tailed.





Using Confidence Limits}
}
*  Alternatively, we can use the confidence interval to make a decision on whether or not we should reject or fail to reject the null hypothesis.
*  If the null value is within the range of the confidence limits, we fail to reject the null hypothesis.
*  If the null value is outside the range of the confidence limits, we reject the null hypothesis.
*  Occasionally a conclusion based on this approach may differ from a conclusion based on the p-value. In such a case, remark upon this discrepancy.




Interpreting P-values}
\textbf{Important}
}
*  From the computer output, determine the $p-$value.  
*  Compare the p-value to $\alpha/k$.  
*  For all of the statistical computing procedures presented here, the tests are all two-tailed, with the exception of the Shapiro-Wilk Test.  
*  If $p-$value $< 0.025$, then we reject $H_0$.  
*  (\textbf{\textit{Shapiro-Wilk Test}}) If $p-$value $< 0.050$, then we reject $H_0$.



