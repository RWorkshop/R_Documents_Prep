%==================================%
\subsection{ \texttt{prop.test()} }


test for proportions. Speciify the number if successes.  The number of trials, and the expected value of the probability under the null hypothesis.

%============================================================================================================%
\subsection{Two-Tailed Test of Population Mean with Known Variance}

The null hypothesis of the two-tailed test of the population mean can be expressed as follows:



where μ0 is a hypothesized value of the true population mean μ.

Let us define the test statistic z in terms of the sample mean, the sample size and thepopulation standard deviation σ :



Then the null hypothesis of the two-tailed test is to be rejected if z ≤−zα∕2 or z ≥ zα∕2 , wherezα∕2 is the 100(1 − α∕2) percentile of the standard normal distribution.

\subsection{Problem}

Suppose the mean weight of King Penguins found in an Antarctic colony last year was 15.4 kg. In a sample of 35 penguins same time this year in the same colony, the mean penguin weight is 14.6 kg. Assume the population standard deviation is 2.5 kg. At .05 significance level, can we reject the null hypothesis that the mean penguin weight does not differ from last year?

\subsection{Solution}

The null hypothesis is that μ = 15.4. We begin with computing the test statistic.

\begin{framed}
<pre><code>
> xbar = 14.6            # sample mean 

> mu0 = 15.4             # hypothesized value 

> sigma = 2.5            # population standard deviation 

> n = 35                 # sample size 

> z = (xbar−mu0)/(sigma/sqrt(n)) 

> z                      # test statistic 

[1] −1.8931
</code></pre>

We then compute the critical values at .05 significance level.

\begin{framed}
<pre><code>
> alpha = .05 

> z.half.alpha = qnorm(1−alpha/2) 

> c(−z.half.alpha, z.half.alpha) 

[1] −1.9600  1.9600
</code></pre>

Answer

The test statistic -1.8931 lies between the critical values -1.9600 and 1.9600. Hence, at .05 significance level, we do not reject the null hypothesis that the mean penguin weight does not differ from last year.

\subsection{Population Mean Between Two Matched Samples}

Two data samples are matched if they come from repeated observations of the same subject. Here, we assume that the data populations follow the normal distribution. Using the paired t-test, we can obtain an interval estimate of the difference of the population means.


Example

In the built-in data set named immer, the barley yield in years 1931 and 1932 of the same field are recorded. The yield data are presented in the data frame columns Y1 and Y2.

\begin{framed}
<pre><code>
> library(MASS)         # load the MASS package 

> head(immer) 

 Loc Var    Y1    Y2 

1  UF   M  81.0  80.7 

2  UF   S 105.4  82.3 

   .....

</code></pre>

Problem

Assuming that the data in immer follows the normal distribution, find the 95% confidence interval estimate of the difference between the mean barley yields between years 1931 and 1932.

Solution

We apply the t.test function to compute the difference in means of the matched samples. As it is a paired test, we set the "paired" argument as TRUE.

> t.test(immer$Y1, immer$Y2, paired=TRUE) 



          Paired t-test 



data:  immer$Y1 and immer$Y2 

t = 3.324, df = 29, p-value = 0.002413 

alternative hypothesis: true difference in means is not equal to 0 

95 percent confidence interval: 

 6.122 25.705 

sample estimates: 

mean of the differences 

                15.913

Answer

Between years 1931 and 1932 in the data set immer, the 95\% confidence interval of the difference in means of the barley yields is the interval between 6.122 and 25.705.

\subsection{Test for variances}


var.test test that two data sets have equal variance.


test that mean is some specified value. 


e.g. Mean =3.50

e.g. 

mean is not equal to 3.50 


The test for equality of variances is important, as many other inference procedures rely on the assumption that the two data sets under consideration have equal variance. Two Sample T test.

\subsection{The paired t test }

The paired t test is another important inference procedure. It takes two sets of paired measurements and tests for a difference. It is called a paired test because each value in one data set  correspond to a value in the other data set. Necessarily there must be equal numbers of elements in both sets.


P values are the probability of a test statistic under the null hypothesis. They are commonly misinterpreted as the probability that a hypothesis is true. Including court cases with tragic consequences.

%============================================================================================%



\subsection{T-test}

The t-test is used to determine statistical differences between two samples. There is also a version that can be used as a paired test i.e. when you have measurements collected as matched pairs.


Attach your data set so that the individual variables are read into memory.

To perform a t-test you type:

\begin{framed}
<pre><code>
> t.test(var1, var2)

Welch Two Sample t-test

data: x1 and x2

t = 4.0369, df = 22.343, p-value = 0.0005376

alternative hypothesis: true difference in means is not equal to 0

95 percent confidence interval:

2.238967 6.961033

sample estimates:

mean of x mean of y

8.733333 4.133333

</code></pre>


This version of the test does not assume that the variance of the two samples is equal and performs a Welch two sample t-test. The "classic" version of the t-test can be run as follows:

\begin{framed}
<pre><code>
> t.test(var1, var2, var.equal=T)

Two Sample t-test

data: x1 and x2

t = 4.0369, df = 28, p-value = 0.0003806

alternative hypothesis: true difference in means is not equal to 0

95 percent confidence interval:

2.265883 6.934117

sample estimates:

mean of x mean of y

8.733333 4.133333

</code></pre>

%=====================================================%


A paired t-test is simple to run; just add \texttt{paired= TRUE} to the basic command. Now the variances of the two samples are considered equal and the basic version is performed. To run a t-test on paired data you add a new term:

\begin{framed}
<pre><code>
> t.test(var1, var2, paired=T)

Paired t-test

data: x1 and x2

t = 4.3246, df = 14, p-value = 0.0006995

alternative hypothesis: true difference in means is not equal to 0

95 percent confidence interval:

2.318620 6.881380

sample estimates:

mean of the differences

4.6

</code></pre>

