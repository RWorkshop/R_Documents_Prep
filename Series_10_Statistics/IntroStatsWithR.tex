
### Outliers}
Another reason that the data may not be normally distributed is the presence of an outlier. We shall look at formal tests for outliers (such as the Grubb's test) next week. Recall that boxplots can be used to detect potential outliers.

## Single Sample Inference Procedures}

While analyzing a (single) sample of data values, we will often want to answer several questions:

* What is the mean value?  (i.e. of 100 roles of a die)
* Is the mean value significantly different from some pre-supposed value?
(i.e. Hypothesis testing ; is the observed mean reasonably close to our expected value)
* What is the level of uncertainty associated with our estimate of the mean value? (i.e. Confidence interval for the estimates)

We refer to the methods used to answer these questions as ***inference procedures***.






A fundamental assumption of linear models is that the
Residuals are normally distributed with mean zero. Investigate this using the RESID command.

The Anderson Darling test is the conventional inference test for assessing whether a data set is normally distibuted. It requires liading the normtest pagkage into the R environment. A nother inference procedure fir testing normality is the Shapiro Wilk test. (We will use this one fir this module, but use Anderson Darling in future).

shapiro.test(x)

The null hypothesis is that the data set is normally distribted. We reject this null hypothesis if the pvalue is too small.

The summary function provides a comprehensive overview of all inference estimates for a model fit, including pvalues for all predictor variables.

The summary output is constructed as a list, and as such, components can be aaccessed using the dollar sign.(Find the names of the component using the names command)

coef(fit1)

returns the intercept and slope estmates for the model.

The adjusted coefficient of determination is computed to account for the presence of more than one predictor variable.
The law of parsimony, the simplest model that adequately explains the outcomes is the best.



